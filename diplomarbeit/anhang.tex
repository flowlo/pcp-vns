\appendix

\chapter{Bisherige Ergebnisse}

Eine Auswahl der bisherigen Ergebnisse unseres Projektes auf aus der wissenschaftlichen Literatur für dieses Optimierungsproblem bekannten Instanzen können der Tabelle~\ref{tab:result} entnommen werden, die Bedeutung der einzelnen Spalten ist wie folgt:

\begin{description}
    \item[Instanz] Name der berechneten Instanz\footnote{Zu finden unter \url{www.ic.uff.br/~celso/grupo/pcp.htm}}.% In dem Format \texttt{n}\textit{(Anzahl der Knoten)}\texttt{}
    \item[$|V|$] Anzahl an Knoten der Instanz.
    \item[$|E|$] Anzahl an Kanten der Instanz.
    \item[$|C|$] Anzahl an Clustern der Instanz.
    \item[$S_i$] Die Anzahl an Farben, die von der durch \emph{onestepCD} berechneten Ausgangslösung benötigt werden.
    \item[$N$] Die verwendeten Nachbarschaften in Reihenfolge der Iteration:
        \begin{description}
            \item[\texttt{c}] \emph{ChangeColor}
            \item[\texttt{n}] \emph{ChangeNode}
            \item[\texttt{d}] \emph{NewVertexColoring-DSATUR}
        \end{description}
    \item[$S_{avg}$] Die im Durchschnitt benötigte Anzahl an Farben der von unserer Implementierung in 30 Testläufen berechneten Lösungen.
    \item[$S_{\sigma}$] Die Standardabweichung der Ergebnisse aller Testläufe (verursacht durch Ran\-dom\-isier\-ung in den einzelnen Verfahren).
    \item[$S_{\Delta}$] Prozentuelle Verbesserung der Ausgangslösung durch den beschriebenen Ansatz.
    \item[$t$] Mittlere Laufzeit in Sekunden (ermittelt auf einem Linux-System Kernel 3.5.0 x86\_64, Intel Core i5-3317U CPU mit 1.70GHz, Compiler gcc 4.7.2 ).
\end{description}

\paragraph{Kurzinterpretation}{
Wie man erkennen kann ist die Reihenfolge, in der das VND die einzelnen Nachbarschaften durchsucht muss um die jeweils besten Resultate zu erzielen, nicht wirklich eindeutig, auch wenn sich eine Tendenz zu \texttt{cnd} abzeichnet. Die erzielten Verbesserung gegenüber der von der Konstruktionsheuristik \emph{onestepCD} ermittelten Ausgangslösungen sind mitunter beachtlich und bewegen sich Großteils im Bereich zwischen 20 und 40\%.}

\begin{table}[!htbp]
\centering
\begin{tabular}{c|rrr|r|c|rrr|r|r}
Instanz & $|V|$ & $|E|$ & $|C|$ & $S_i$ & $N$ & $S_{min}$ & $S_{avg}$ & $S_{\sigma}$ & $S_{\Delta}$ & $t$ \\
\hline\hline
\texttt{n90p5t2s1.pcp} & 90	& 2019	& 45 & 13 & \texttt{cdn} & 9 & 9.61 & 1.8 & 44.44 & 693.2\\
\texttt{n90p5t2s2.pcp} & 90	& 1963	& 45 & 11 & \texttt{ncd} & 8 & 9.29 & 1.8 & 37.50 & 550.6\\
\texttt{n90p5t2s3.pcp} & 90	& 2045	& 45 & 13 & \texttt{cdn} & 9 & 9.61 & 1.8 & 44.44 & 568.4\\
\texttt{n90p5t2s4.pcp} & 90	& 2014	& 45 & 12 & \texttt{cdn} & 9 & 9.61 & 1.8 & 33.33 & 593.9\\
\texttt{n90p5t2s5.pcp} & 90	& 2057	& 45 & 13 & \texttt{dnc} & 9 & 9.87 & 1.9 & 44.44 & 789.7\\
\texttt{n90p7t2s4.pcp} & 90	& 2821	& 45 & 16 & \texttt{ndc} & 12 & 13.35 & 2.5 & 33.33 & 656.1\\
\texttt{n90p7t2s5.pcp} & 90	& 2834	& 45 & 16 & \texttt{cdn} & 13 & 13.94 & 2.6 & 23.08 & 847.7\\
\texttt{n90p8t2s1.pcp} & 90	& 3257	& 45 & 19 & \texttt{dcn} & 15 & 16.77 & 3.2 & 26.67 & 839.0\\
\texttt{n90p8t2s2.pcp} & 90	& 3188	& 45 & 19 & \texttt{cdn} & 15 & 16.06 & 3.0 & 26.67 & 1125.5\\
\texttt{n90p9t2s1.pcp} & 90	& 3631	& 45	& 24 & \texttt{cnd} & 19 & 20.48 & 3.8 & 26.32 & 1247.4\\
\texttt{n90p9t2s2.pcp} & 90	& 3614	& 45	& 23 & \texttt{cnd} & 18 & 20.42 & 3.9 & 27.78 & 1051.3\\
\texttt{n90p9t2s3.pcp} & 90	& 3615	& 45	& 24 & \texttt{cdn} & 19 & 19.97 & 3.7 & 26.32 & 1432.6\\
\texttt{n90p9t2s4.pcp} & 90	& 3619	& 45	& 23 & \texttt{dcn} & 17 & 19.77 & 3.8 & 35.29 & 1407.4\\
\texttt{n90p9t2s5.pcp} & 90	& 3634	& 45	& 24 & \texttt{dcn} & 18 & 19.42 & 3.6 & 33.33 & 1357.1\\
\texttt{n100p5t2s1.pcp} & 100	& 2494	& 50	& 13 & \texttt{cdn} & 10 & 10.58 & 2.0 & 30.00 & 689.7\\
\texttt{n100p5t2s2.pcp} & 100	& 2428	& 50	& 12 & \texttt{cdn} & 10 & 10.55 & 1.9 & 20.00 & 489.0\\
\texttt{n100p5t2s3.pcp} & 100	& 2513	& 50	& 12 & \texttt{ndc} & 9 & 9.94 & 1.9 & 33.33 & 610.6\\
\texttt{n100p5t2s4.pcp} & 100	& 2442	& 50	& 12 & \texttt{cdn} & 9 & 9.94 & 1.9 & 33.33 & 602.3\\
\texttt{n100p5t2s5.pcp} & 100	& 2500	& 50	& 12 & \texttt{dnc} & 9 & 10.55 & 2.1 & 33.33 & 848.7\\
\texttt{n120p5t2s1.pcp} & 120	& 3593	& 60	& 14 & \texttt{cdn} & 11 & 12.03 & 2.3 & 27.27 & 1549.0\\
\texttt{n120p5t2s2.pcp} & 120	& 3544	& 60	& 14 & \texttt{cdn} & 11 & 11.77 & 2.2 & 27.27 & 906.8\\
\texttt{n120p5t2s3.pcp} & 120	& 3613	& 60	& 14 & \texttt{cdn} & 12 & 11.90 & 2.2 & 16.67 & 651.0\\
\texttt{n120p5t2s4.pcp} & 120	& 3536	& 60	& 14 & \texttt{dcn} & 11 & 11.87 & 2.2 & 27.27 & 1334.2\\
\texttt{n120p5t2s5.pcp} & 120	& 3623	& 60	& 15 & \texttt{ndc} & 11 & 12.23 & 2.3 & 36.36 & 1050.3\\
\end{tabular}
\caption{Ergebnisse der VNS im Überblick}
\label{tab:result}
\end{table}
