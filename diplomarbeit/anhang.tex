\appendix

\chapter{Ergebnisse}

Mit Ende der Arbeit an dem Projekt im Rahmen der Diplomarbeit wurden in Tabelle~\ref{tab:result} ersichtliche Ergebnisse auf dem Berechnungsnetzwerk der Arbeitsgruppe für Algorithmen und Datenstrukturen des Instituts für Computergraphik und Algorithmen an der Technischen Universität Wien erzielt.

Die Instanzen stammen von der genannten Arbeitsgruppe und sind mit jenen von \cite{Noronha2006} vergleichbar.

Das Abbruchkriterium der VNS für diese Ergebnisse war ein Zeitlimit von 600s oder 5000 fehlgeschlagene \enquote{Schüttelversuche}.


\subsection*{Legende}

\begin{description}
    \item[Instanz] Name der berechneten Instanz.
    \item[$|V|$] Anzahl der Knoten der Instanz.
    \item[$|E|$] Anzahl der Kanten der Instanz.
    \item[$|C|$] Anzahl der Clustern der Instanz.
    \item[$S_i$] Die Anzahl der Farben, die von der durch \emph{onestepCD} berechneten Ausgangslösung benötigt werden.
    \item[$N$] Die verwendeten Nachbarschaften in Reihenfolge der Iteration:
        \begin{description}
            \item[\texttt{c}] \emph{ChangeColor}
            \item[\texttt{n}] \emph{ChangeNode}
            \item[\texttt{a}] \emph{ChangeAll}
            \item[\texttt{d}] \emph{DSATUR}
        \end{description}
    \item[$S_{avg}$] Die im Durchschnitt benötigte Anzahl an Farben der von unserer Implementierung in 40 Testläufen berechneten Lösungen.
    \item[$S_{\sigma}$] Die Standardabweichung der Ergebnisse aller 40 Testläufe (verursacht durch Ran\-dom\-isier\-ung in den einzelnen Verfahren).
    \item[$S_{\Delta}$] Prozentuelle Verbesserung der Ausgangslösung durch den beschriebenen Ansatz.
    \item[$t$] Die mittlere Laufzeit in Sekunden.
\end{description}

\begin{table}[p]
\centering
\begin{tabular}{|c|ccc|c|c|ccc|c|c|}
\hline
Instanz & \multicolumn{1}{c}{$|V|$} & \multicolumn{1}{c}{$|E|$} & \multicolumn{1}{c|}{$|C|$} & \multicolumn{1}{c|}{$S_i$} & \multicolumn{1}{c|}{$N$} & $S_{min}$ & \multicolumn{1}{c}{$S_{avg}$} & \multicolumn{1}{c|}{$S_{\sigma}$} & \multicolumn{1}{c|}{$S_{\Delta}$} & \multicolumn{1}{c|}{$t$} \\
\hline\hline

\multirow{3}{*}{\texttt{dsjc500.5-1.in}} & \multirow{3}{*}{500} & \multirow{3}{*}{62624} & \multirow{3}{*}{500} & \multirow{32}{*}{69} &
      \texttt{cnad}	& \multirow{3}{*}{62}	& 63.67	& 0.52	& \multirow{3}{*}{11.29}	& 494109.3\\
&&&&& \texttt{cand}	&			& 63.55	& 0.55	&				& 494685.3\\
&&&&& \texttt{acnd}	&			& 63.67	& 0.52	&				& 492537.0\\
\cline{1-4}\cline{6-11}
\multirow{14}{*}{\texttt{dsjc500.5-2.in}} & \multirow{14}{*}{1000} & \multirow{14}{*}{249671} & \multirow{14}{*}{500} & &
      \texttt{cnda}	& \multirow{14}{*}{60}	& 61.30	& 0.51	& \multirow{14}{*}{15.00}	& 482239.0\\
&&&&& \texttt{cadn}	&			& 61.27	& 0.50	&				& 472219.3\\
&&&&& \texttt{dcan}	&			& 61.55	& 0.55	&				& 488494.5\\
&&&&& \texttt{danc}	&			& 61.40	& 0.54	&				& 497711.0\\
&&&&& \texttt{ncda}	&			& 61.13	& 0.40	&				& 496069.8\\
&&&&& \texttt{ndca}	&			& 61.27	& 0.50	&				& 499308.0\\
&&&&& \texttt{ndac}	&			& 61.17	& 0.49	&				& 508483.5\\
&&&&& \texttt{nacd}	&			& 61.45	& 0.55	&				& 448795.8\\
&&&&& \texttt{nadc}	&			& 60.95	& 0.31	&				& 499959.3\\
&&&&& \texttt{acdn}	&			& 60.88	& 0.40	&				& 489919.3\\
&&&&& \texttt{adcn}	&			& 60.80	& 0.40	&				& 502662.8\\
&&&&& \texttt{adnc}	&			& 60.95	& 0.31	&				& 509733.8\\
&&&&& \texttt{ancd}	&			& 61.70	& 0.51	&				& 443322.5\\
&&&&& \texttt{andc}	&			& 61.00	& 0.45	&				& 499170.8\\
\cline{1-4}\cline{6-11}
\multirow{14}{*}{\texttt{dsjc500.5-3.in}} & \multirow{14}{*}{1500} & \multirow{14}{*}{562401} & \multirow{14}{*}{500} & &
      \texttt{cdna}	& \multirow{14}{*}{60}	& 61.20	& 0.46	& \multirow{14}{*}{15.00}	& 477578.0\\
&&&&& \texttt{cadn}	&			& 61.20	& 0.51	&				& 465780.5\\
&&&&& \texttt{dcna}	&			& 61.38	& 0.53	&				& 479848.8\\
&&&&& \texttt{dnac}	&			& 61.23	& 0.52	&				& 498028.5\\
&&&&& \texttt{danc}	&			& 61.23	& 0.47	&				& 489702.3\\
&&&&& \texttt{ncda}	&			& 60.95	& 0.22	&				& 475189.8\\
&&&&& \texttt{ndca}	&			& 60.88	& 0.33	&				& 484062.0\\
&&&&& \texttt{ndac}	&			& 60.98	& 0.16	&				& 488796.3\\
&&&&& \texttt{nacd}	&			& 61.55	& 0.59	&				& 432441.0\\
&&&&& \texttt{nadc}	&			& 60.92	& 0.35	&				& 491085.8\\
&&&&& \texttt{acdn}	&			& 60.95	& 0.22	&				& 483350.0\\
&&&&& \texttt{adcn}	&			& 60.98	& 0.35	&				& 493467.8\\
&&&&& \texttt{adnc}	&			& 60.95	& 0.22	&				& 501021.3\\
&&&&& \texttt{andc}	&			& 60.98	& 0.27	&				& 491790.3\\
\cline{1-4}\cline{6-11}
\texttt{dsjc500.5-4.in} & 2000 & 999508 & 500 & & \texttt{adnc} & 59 & 60.65 & 0.53 & 16.95 & 497422.8\\
\hline
\end{tabular}
\caption{Ergebnisse der VNS}
\label{tab:result}
\end{table}

\section{Kurzinterpretation}
Wie man erkennen kann ist die Reihenfolge, in der das VND die einzelnen Nachbarschaften durchsuchen muss, um die jeweils besten Resultate zu erzielen, nicht wirklich eindeutig. Die erzielten Verbesserung gegenüber der von der Konstruktionsheuristik \emph{onestepCD} ermittelten Ausgangslösungen sind mitunter beachtlich und bewegen sich großteils im Bereich zwischen 11 und 15\%.
