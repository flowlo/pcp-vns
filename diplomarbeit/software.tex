\section{Implementierung}
% TODO: everything 

\subsection{C++}
Als primäre Programmiersprache wurde C++ gewählt. Verschiedene Gründe sprachen für C++, unter anderem die an C angelehnte Syntax, aber
auch die gute Performance im Vergleich zu Sprachen, welche in einer Virtuellen Maschine laufen, oder gar interpretiert werden. Zu 
all diesen Gründen kam noch hinzu, dass für C++ bereits ausgezeichnete Codebibliotheken zur Verfügung stehen, welche ein schnelles
und effizientes Arbeiten ermöglichen.

Bei C++ handelt es sich um eine im Jahr 1979 von Bjanre Stroustrup entwickelte Programmiersprache, welche anfangs als \textit{C mit Klassen}
ausgelegt wurde. Über mehrere Entwicklungsschritte entwickelte sich eine Sprache, dessen Wurzeln in der Sprache C immer noch zu erkennen sind, 
dessen Möglichkeiten jene von C aber bei weitem übersteigen. 

Wie C handelt es sich bei C++ um eine statisch-typisierte Programmiersprache, das heißt der Compiler kann schon während der Übersetzung die
syntaktisch korrekte Verwendung von Datentypen sicherstellen. Dadurch können solche Überprüfungen während der Ausführung des Programmes
entfallen, was zu einer enormen Beschleunigung der Ausführung des Programmes führt. Außerdem ermöglicht es dem Entwickler Fehler mit
inkorrekt verwendeten Datentypen einfacher zu erkennen, da eine Fehlermeldung des Übersetzers eine genau Zeile angeben kann, in der 
der Fehler verursacht wird. 

Obwohl sich C++ wesentlich von C unterscheidet, ist es immer noch kompatibel mit C. Eine Stück C Code kann immer noch in C++ verwendet und
übersetzt werden. Dadurch können nicht nur Code-Bibliotheken aus dem C++-Umfeld verwendet werden, sondern auch ursprünglich für C gedachte
Bibliotheken eingebunden und verwendet werden. Da C immer noch eine der beliebtesten Programmiersprachen weltweit, und vor allem \textbf{die}
Sprache für Systemprogrammierung und performante Software ist, bedeutet diese Mitnahme der Vorteile aus der C-Welt in die Welt von C++
einen großen Vorteil bei der Programmierung.

Des Weiteren wurde C++ möglichst plattformunabhängig designt. Während manche Funktionen, welche vom Betriebssystem bereit gestellt werden, 
natürlich nicht wirklich plattformunabhängig sein können, wurde in C++ versucht, keine Funktionen oder Eigenschaften zu verwenden, welche
nur von einem bestimmten System ausgeführt werden können. Aus diesem Grund sind auch die meisten Code-Bibliotheken für C++ für die meisten
Plattformen erhältlich. 

Obwohl C++ wesentlich mehr Funktionen als C bietet, ist trotzdem noch die Anlehnung an C zu erkennen. Daher ist es auch nicht verwunderlich das C++-Programme meist nur geringfügig langsamer sind, als 
vergleichbare C-Programme. In manchen Fällen ist C++ sogar schneller als C, da der Compiler andere, und umständen bessere, Optimierungen vornehmen kann. 

\subsubsection{STL}
\label{sec:stl}
Zusätzlich zu erwähnen ist die umfangreiche \textit{Standard Template Library (STL)}, welche es inzwischen sogar fast vollständig in den 
C++-Standard gebracht hat. Bei der STL handelt es sich um eine von den meisten Compilern bereitgestellte Code-Bibliothek für häufig 
verwendete Datenstrukturen und Hilfsroutinen, welche eine wesentliche Erleichterung für jeden C++-Programmierer darstellen. Unter anderem
in der STL enthaltene Datenstrukturen sind Listen, Stacks, aber auch Iteratoren für die meisten Datenstrukturen, häufig verwendete Algorithmen
wie Quicksort, Binarysearch. Hinzu kommen vereinheitlichte Ein- und Ausgabemöglichkeiten für verschiedenste Plattformen, sowie die 
Möglichkeit der parallelen Ausführung von Software mit Hilfe von Threads.

Die STL wurde inzwischen in den Sprachstandard von C++ aufgenommen, so dass jeder standardkonforme Comipler eine Implementierung der STL inkludiert. Mit der STL hebt sich C++ stark von C ab, für 
das es keine solche allgemein einsetzbare Codebibliothek gibt. Der Preis für diese Einfachheit ist die Zeit, die ein einzelner Übersetzungsdurchgang benötigt. Da die STL stark auf die namensgebenden
\textit{Templates} setzt, welche es ermöglichen eine einzelne Datenstruktur für eine Vielzahl an Datentypen verfügbar zu machen, muss der Compiler diese \textit{Templates} zum Übersetzungszeitpunkt
auflösen, ein Vorgang, welcher einen stark negativen Einfluss auf die Übersetzungszeit hat.

\subsection{Python}
% TODO: LOLO!

\subsection{Make}
\label{sec:make}
Bei GNU make handelt es sich um ein Standardwerkzeug der Linux-Software-Entwicklung. Mit Hilfe von make kann die Übersetzung des Quellcodes in den Programmcode automatisiert werden. 
Dabei stellt make viele verschiedene Konfigurationsmöglichkeiten bereit. Unter anderem können verschiedene Compiler, aber auch verschiedene Übersetzungsmodi eingestellt werden.
Um ein schnelleres Übersetzen zu ermöglichen, werden außerdem nur jene Quellcode-Dateien neu übersetzt, welche sich seit dem letzten Mal verändert haben. 

In der Konfigurationsdatei für make, dem so genannten Makefile, können unter anderem mehrere \textit{Targets} angegeben werden, also quasi verschiedene Endprogramme, oder verschiedene
Übersetzerkonfigurationen für das selbe Programm. Zu diesen \textit{Targets} werden dann Ab\-hängig\-keiten definiert, welche bestimmen, in welcher Reihenfolge das Programm übersetzt werden 
muss. Impliziert wird durch diese Abhängigkeitsdefinition auch, wann ein Programmteil neu übersetzt werden muss. Ist eine der Abhängigkeiten einer bestimmten Enddatei jünger als die 
Enddatei selber, muss diese neu übersetzt werden. Durch eine geschickte Aufteilung des Quellcodes auf mehrere Dateien kann so eine komplette Neuübersetzung vermieden werden. 

Durch verschiedene Variablen, welche ebenfalls im Makefile spezifiziert sind, kann außerdem, ohne großen Aufwand, der Übersetzer, welcher die eigentliche Hauptaufgabe übernimmt ausgetauscht werden.
Die für diese Arbeit verwendeten Compiler werden in Abschnitt \ref{sec:compiler} besprochen. Vereinfacht wird dies durch die beinahe gleiche Befehlssyntax, die von beiden Compilern verwendet wird.
Hinzu kommt die Tatsache, dass die von den beiden Compilern erzeugten Dateiformate miteinander kompatibel sind, so dass auch uneinheitlicher Übersetzung mit immer wieder wechselnden Übersetzern
keine Fehler entstehen.

Außerdem hilfreich ist, dass make auch Skripte oder kleinere Kommandos automatisch aus\-füh\-ren kann. Häufig verwendet ist etwa die Routine zur Bereinigung des Programmverzeichnisses von bei der
Übersetzung entstandenen temporären Dateien, die nachher nicht mehr benötigt wurden. Auch nützlich ist etwa die vollständige Entfernung jeder Spur einer Compilierung, um eine gänzlich neue
Übersetzung einzuleiten. Auch kann mit Hilfe von solchen Skripten das entstandene Programm nach Übersetzung automatisch auf eine Testmaschine geschickt werden, um die korrekte Funktion
des Programmes zu verifizieren.

\subsection{Compiler}
\label{sec:compiler}
Der Compiler, oder Übersetzer, ist dafür verantwortlich, ein Programm von einer Quellsprache, in eine Zielsprache zu übersetzen. Im Falle dieser Arbeit war die Quellsprache C++, und die
Zielsprache die Maschinensprache des Prozessors, also ein x86-Assembly. Übersetzer sind außerordentlich komplexe Stücke an Software, welche nicht einfach die Befehle eines Programmierers eins zu
eins übernehmen, sondern auch noch Optimierungen vornehmen, um eine beschleunigte Programmausführung zu ermöglichen. Da dieser Vorgang sehr zeitaufwendig sein kann, ist es wichtig, Werkzeuge wie
das in Abschnitt \ref{sec:make} besprochene make einzusetzen, um eine vollständige Neuübersetzung bei nur marginalen Änderungen im Quellcode zu verhindern. 

Da C++ eine besonders komplexe Sprache, mit vielen verschiedenen, aufwendigen Funktionen und Sprachbesonderheiten ist, ist die Übersetzung eines C++-Quellcodes besonders aufwendig. Selbst kleinere
C++-Programme können schon mehrere Minuten zum vollständigen Übersetzen brauchen. Dies gilt insbesondere für Quellcode, welche die C++-Funktion der so genannten \textit{Templates} benützt. \textit{Templates}
erlauben es, neben anderen Dingen, eine allgemeine Datenstruktur für viele verschiedene Datentypen zu programmieren, ohne eine konkrete Implementation für einen bestimmten Datentyp notwendig zu machen.
Da auch mehrere \textit{Templates} geschachtelt werden können, und dies alles zum Zeitpunkt der Übersetzung expandiert werden muss, um schließlich übersetzt werden zu können, kann bei starker Verwendung
von solchen Funktionen die Geschwindigkeit der Übersetzung stark beeinträchtigt werden.

Um diese Probleme mit \textit{Templates} elegant zu umgehen, gibt es \textit{Precompiled Headers}. Normalerweise werden Header-Dateien, welche allgemeine Deklarationen von Klassen, verwendeten Datentypen
und ähnliches enthalten, zum Übersetzungszeitpunkt in den C++-Quellcode, welcher übersetzt werden soll, komplett kopiert, und diese Gesamteinheit als eine Über\-setzungs\-einheit angesehen. Dadurch muss, sollte
sich ein Teil der Definitionen innerhalb des C++-Codes ändern, auch die Definition neu übersetzt werden. Mit \textit{Precompiled Headers} wird dieses Problem gelöst, in dem schon der Header als zu übersetzende
Einheit angesehen wird, und als vom eigentlichen Quelltext abgetrennte Einheit angesehen wird. Der Header, welcher ja unter anderem auch die Deklarationen für \textit{Templates} enthält, muss nur noch
übersetzt werden, wenn sich in seinem eigenen Code etwas ändert. Dadurch können die eigentlichen Programmabschnitte, welche in normalen C++-Quelldateien zu finden sind, wesentlich schneller
compiliert werden, ohne auf die Vorteile von \textit{Templates} zu verzichten.


\subsubsection{GCC}
GCC, oder auch GNU Compiler Collection, ist ein Sammlung an Übersetzern welche alle unter der Open-Source-Lizenz GPL veröffentlich wurden. GCC wird häufig als der Standard-Compiler unter Linux angesehen,
und erfreut sich immer noch großer Beliebtheit. Die Sammlung unterstützt viele verschiedene Sprachen, unter ihnen auch C++, sowie mehrere verschiedene Standards einzelner Sprachen. Auf Grund seiner
ausgiebigen Testung und Erprobung im alltäglichen Einsatz, und nicht zuletzt seiner freien Verfügbarkeit, bleibt GCC immer noch einer beliebtesten Compiler weltweit. Dabei ist die Sammlung verschiedener
Compiler vor allem im Umfeld der Softwareentwicklung für Linux beliebt, er kann aber auch für die Entwicklung auf Windows oder anderen Betriebssystemen eingesetzt werden.

Ein wichtiges Merkmal von \textit{g++}, dem C++-Compiler in der Sammlung von GCC, ist sein evolutionäres Wachstum seit seiner Entstehung. Da C++ mit den Jahren immer wieder leicht angepasst und um Funktionen
erweitert wurde, musste sich auch GCC anpassen, um mit diesen Änderungen Schritt zu halten. Daher wurden im Laufe der Jahre immer wieder Teile von \textit{g++} neu geschrieben, umgebaut, oder gänzlich
neue Funktionen hinzuprogrammiert. Daraus ergeben sich aber Konsequenzen, welche unter anderem die Code-Qualität negativ beeinträchtigen. Zwar Implementiert GCC auch den neuesten Standard von C++, allerdings
ist die Qualität der Implementierung sehr stark abhängig von dem Alter des entsprechenden Standards. Während ältere Standards bereits ausgiebig getestet wurden, ist bei neuen Standards eine korrekte 
Implementierung nicht immer gegeben. 

Auf Grund dieser gewachsenen Struktur von GCC ist der \textit{g++}-Compiler nicht der schnelleste Vertreter seiner Art. Vor allem die Übersetzung von Headern nimmt mit der GCC mehr Zeit in Anspruch als
bei der Konkurrenz. Wie viele andere Compiler unterstützt auch \textit{g++} die übersetzung von Header-Dateien in \textit{Precompiled Headers}. Dazu wird ein GCC-eigene Format verwendet, und \textit{g++} sucht bei einer
späteren Einbindung eines Headers in den Quellcode nach einer gleichnamigen Datei mit der Endung ``.gch''. Die Verwendung von solchen \textit{Precompiled Headers} beschleunigt die Übersetzung ungemein, 
obwohl die Übersetzung des Headers selber einige Zeit in Anspruch nimmt. Die für diese Arbeit verwendeten \textit{Precompiled Headers} bewegeten sich in der Größenordnung von knapp über 100 Megabyte, 
eine erstaunliche Datenmenge für eine einzige Übersetzungseinheit. Diese Datenmengen sind vor allem den \textit{Templates} geschuldet, welche während ihrer Expansion wesentlich an Speicherbedarf zunehmen.

\subsubsection{clang}
Bei clang handelt es sich um einen neuen Konkurrenten für GCC am Markt der Open-Source-Compiler. Wie bei GCC handelt es sich bei clang nicht um einen einzelnen Übersetzer für eine einzelne Programmiersprache,
sondern viel mehr um eine Sammlung an Übersetzern für vor allem mit C verwandten Sprachen wie C++ und Objective-C. Anders als GCC ist clang aber nicht über mehrere Jahrzehnte angewachsen, das Projekt
startete erst vor wenigen Jahren. Auch baut clang nicht von Grund auf einen neuen Compiler, sondern benützt das Rahmenwerk der LLVM, der \textit{Low Level Virtual Machine}. LLVM ist ein Projekt, welches
mit dem Ziel gestartet wurde, Techniken des Just-in-Time-Übersetzens zu den Sprachen C und anderen Low-Level-Sprachen zu bringen. Dazu wird eine eindeutige Zwischensprache, zwischen Quell- und Zielsprache
definiert, welche dann von LLVM optimiert werden kann. Diese Technik, welche ursprünglich zur Laufzeit eines Programmes zum Einsatz kommen sollte, wurde schließlich so adaptiert, um einen Compiler von
eben jener Zwischensprache zur Zielsprache zu bilden. 

Clang wurde nun mit dem Ziel geschaffen, einen sauber programmierten, standard-konformen Compiler für den produktiven Einsatz zu schaffen. Dazu wurde zunächst vor allem auf die Entwicklung eines C-Frontends
Wert gelegt. Als das C-Frontend einen arbeitsfähigen Zustand erreichte, beschloss man, auch C++ als Quellsprache mit einzuschließen. Da C++ eine sehr umfangreiche Sprache mit vielen verschiedenen Spracheigenschaften 
ist, schreitet die C++-Implementierung wesentlich langsamer voran als jene des C-Zweiges. Entzwischen hat aber auch dieser Entwicklungszweig einen produktiven Zustand erreicht. 

Im Vergleich mit GCC bietet clang vor allem den Vorteil der Geschwindigkeit. Ein einzelner Übersetzerlauf dieser Arbeit ist um durchschnittlich 20\% schneller als mit GCC, und die erzeugten Object-Dateien
sind um durchschnittlich 30\% kleiner. Dieser Vorteil ist wohl der neueren Programmierung, ohne Mitnahme von Altlasten geschuldet. Der Preis für diesen Vorteil ist eine etwas langsamere Programmausführung
des fertig übersetzten Programmes, da unter anderem GCC einige Optimierungsmöglichkeiten nutzt, welche LLVM noch unbekannt sind. Im Allgemeinen sind die Unterschiede zwischen den beiden Compilern aber
vernachläßigbar gering.

\subsection{DDD}
DDD steht für Data Display Debugger, und ist ein weiteres Werkzeug zur Softwareentwicklung von GNU. DDD stellt Werkzeuge zum Debuggen, also der Fehlersuche in Software, zur Verfügung. Um produktiv eingesetzt
werden zu können muss eine spezielle Option während der Übersetzung mit GCC gesetzt werden, damit DDD einzelnen Programmschritten den richtigen Quellcodezeilen zuordnen kann. DDD ist eine sehr vielseitige
Anwendung mit graphischer Benutzeroberfläche, welche sehr intuitiv und einfach zu bedienen ist, und trotzdem viele Optionen bietet.

Eines der wichtigesten Werkzeuge zur Fehlersuche innerhalb eines laufenden Programmes sind Unterbrechungspunkte, also eine bestimmte Stelle innerhalb des Quellcodes, an dem die Ausführung angehalten werden 
soll, um den genauen Zustand des Programmes zu untersuchen. DDD stellt sowohl normale Unterbrechungspunkte, als auch spezielle Punkte, welche nur unter vom Benutzer definierten Bedingungen auslösen, bereit.
Sobald ein solcher Unterbrechungspunkt erreicht ist, kann DDD die Belegung sämtlicher Variablen anzeigen und vergleichen. Dies kann besonders nützlich sein, um ungeplantes Verhalten innerhalb des
Progamm\-ablaufes zu untersuchen, und zu beheben. 

Außerdem bietet DDD die Möglichkeit, im Falle eines Programmabsturzes die auslösende Zeile des Quellcodes zu ermitteln, und von dort aus auf die Ursachen des Absturzes zu schließen. Dies ist besonders
praktisch, da eine genaue Eingrenzung der Absturzursache ansonsten äußerst schwierig ist. Sollte ein Programm schadhaftes, oder fehlerhaftes Verhalten an den Tag legen, zum Beispiel unberechtigen Speicherzugriff
wird normalerweise das Programm sofort vom Betriebssystem beendet. DDD protokolliert aber jede aufgerufenen Zeile Code mit, und kann daher genaue Auskunft geben, in welchem Programmabschnitt der
Fehler aufgetreten ist.

Ein weiteres praktisches Werkzeug von DDD ist die Möglichkeit, Pointer, welche auf vom Programm reservierte Speicherbereiche zeigen, genauenstens zu verfolgen. Durch eine übersichtliche graphische Darstellung
ist es einfach, zu vergleichen welche Pointer auf ein und den selben Speicherbereich zeigen, was unter Umständen gewollt sein kann, oder aber auch einen fehlerhaften Programmablauf hervorruft.

\subsection{Boost}
Boost, oder auch Boost C++-Libraries, ist eine Codebibliothek, welche versucht, Un\-zu\-läng\-lich\-keiten im Sprachstandard von C++ geschickt zu umgehen, und Benutzern ein einfacheres Programmieren zu ermöglichen.
Boost besteht aus mehreren unabhängigen Unterbibliotheken, welche zum größten Teil unter der Boost Software Lizenz stehen, welche sowohl eine Open-Source, als auch eine kommerzielle Nutzung zulässt. Alle
Unterbibliotheken sind portabel designt, um eine Nutzung auf allen Betriebssystemen mit C++-Compiler zu ermöglichen. 

Die Entwicklung von Boost began im Jahr 2000, als Mitglieder des C++\--Standardisierungs\-komitees zusammentraten um Erweiterungen des C++-Standards auf einer breiteren öffentlichen Plattform zu präsentieren.
Im Laufe der Entwicklung wurden immer wieder auch ehemals kommerzielle Produkte von Firmen in Boost eingegliedert, wie etwa die Graphikbibliothek GIL, welche ursprünglich von Adobe stammt, inzwischen
aber in den Boost-Stamm aufgenommen wurde.

Boost erweitert den Umfang an vorprogrammierten Datenstrukturen, und auch Algorithmen, und fügt bestehende Strukturen nahtlos ein. Dabei versucht Boost immer eine möglichst einfache und klar verständliche
Schnittstelle zu bieten, welche transparent mit Datenstrukturen etwa aus der STL umgeht, welche in Abschnitt \ref{sec:stl} besprochen werden. Von Boost geschaffene Datenstrukturen sind sehr ähnlich zu jenen
aus der STL zu verwenden, und wie jene bieten sie auch einfache Möglichkeiten des Zugriffes auf Datenelemente über Iteratoren. 

Des Weiteren bietet Boost neue Möglichkeiten im Umgang mit Strings, also Zeichenketten, sowie weitere nützliche Hilfsfunktionen, welche von den meisten Programmen häufig gebraucht werden. Außerdem bietet
Boost Algorithmen und Datenstrukturen für mathematische Berechnungen, die über triviale Aufgaben wie addieren oder subtrahieren von einfachen Zahlen hinausgehen. So bietet Boost eine Unterbibliothek für 
Statistik, und auch für Graphen, welche in dieser Arbeit zum Einsatz kam.

Außerdem wurden in dieser Arbeit viele Hilfsroutinen, welche von Boost bereitgestellt wurden verwendet, und mehrere Datenstrukturen aus Boost-Unterbibliotheken kamen ebenfalls zum Einsatz.

\subsubsection{graph}
Da es sich bei dem PCP um ein Optimierungsproblem aus dem Umfeld der Graphentheorie handelt, ist natürlich eine Graphenstruktur innerhalb des Programmes unerlässlich. Boost bietet eine ganze Unterbibliothek, 
welche rein graphenbezogene Datenstrukturen und Algorithmen bereitstellt. Diese Graphenbibliothek ist gut getestet, vielseitung, stark anpassbar, schnell und einfach einzusetzen und war daher die ideale Wahl
für die Lösung des PCP.

Boost bietet zwei grundsätzlich Unterschiedliche Implementierungen eines Graphen an. Bei der ersten Variante werden alle Kanten in einer Liste, oder ähnlichen Datenstruktur gespeichert, bei
der zweiten Variante wird eine Matrix verwendet, um die Beziehunen zwischen zwei Knoten darzustellen. Während die zweite Variante für sehr Dichte Graphen einige Vorteile bietet, ist die gesamte
Datenstruktur bei weitem nicht so flexibel wie die Variante mit einer Listenstruktur. In der von Boost verwendeten Implementierung dieser Adjazenzmatrix werden viele praktischen Funktionen um einen
Graphen zu Manipulieren nicht unterstützt. So ist es zum Beispiel nicht möglich einen Knoten nachträglich wieder zu entfernen. 

Daher viel die Wahl auf eine Adjazenzliste, welche den Graphen für diese Arbeit bereitstellte. Auch hier gibt es mehrere verschiedene Konfigurationsmöglichkeiten, etwa die genaue Art der Speicherung von
Knoten und Kanten. Für die Speicherung der Kanten wurde ein Vektor gewählt, ein zusammenhängender Speicherbereich, welche bei Bedarf vergößert oder verkleinert werden kann. Dies bietet mehrer Vorteile, etwa
einen direkten Zugriff auf einen Knoten über einen Index, welcher in Form einer einfach ganzen Zahl leicht zu benützen war. Dies bedeutet, das zu jedem Zeitpunkt direkt auf einen bestimmten Knoten zugegriffen
werden kann. Da die Anzahl der Knoten quasi immer Konstant blieb, mit Ausnahme des erstmaligen Aufbaus des Problemgraphens und dem Errechnen der Initiallösung, werden die Nachteile einer vektorbasierten
Speicherung der Knoten bei weitem von den Vorteilen aufgewogen. 

Da im Rahmen der Variablen Nachbarschaftssuche mehrere Nachbarschaften immer wieder eine Veränderung der Adjazenzen vornehmen, viel die Wahl der Datenstruktur für die Speicherung der Kanten zunächst auf eine
Liste. Zu jedem Knoten wird eine Liste der Nachbarknoten gespeichert, woraus dann die Kanten gebildet werden. Da das Löschen aus, und das Einfügen in eine Liste in konstanter Laufzeit möglich ist, wurden
mit der Listenstruktur die besten Ergebnisse erwartet. Ausgiebige Tests ergaben jedoch, dass auf für die Speicherung der Kanten ein Vektor besser geeignet war. Da beim Löschen zumeist gleich alle
Kanten gelöscht werden, reduzierte sich der Vorteil einer Listenstruktur, und Einfügen ist auch in einen Vektor in zumeist konstanter Zeit möglich. 

Eine weitere Konfigurationemöglichkeit betrifft das Verhalten des Graphen beim Hinzufügen von Kanten. Es gibt gerichtete und ungerichtete Graphen, wobei gerichtete Graphen Kanten besitzen, welche quasi
als Einbahn fungieren, sie haben einen Anfangsknoten und einen Zielknoten, und stellen keine Verbindung vom Zielknoten zum Anfangsknoten her. Im Gegensatz dazu stehen die ungerichtete Graphen, bei denen
eine Kante automatisch eine Beziehung in beide Richtungen darstellt. Auf Grund der Eigenschaften des PCP fiel die Wahl auf einen ungerichteten Graphen, da kein Bedarf an gerichteten Kanten bestand.

Eine weitere von Boost bereitgestellte Funktion des Graphen ist die Speicherung von Eigenschaften im direkten Zusammenhang mit Kanten und Knoten. Da jeder Knoten im PCP zu einer Partition zugeteilt wird, kann
über diesen Mechanismum einfach und schnell von einem Knoten auf die Partition geschlossen werden. Diese Eigenschaften können außerdem dazu eingesetzt werden, den originalen Index eines Knotens zu speichern.
Da Boost bei der Entfernung eines Knotens automatisch den Index der nachfolgenden Knoten um eins dekrementiert, kann normalerweise nicht mehr auf den Index des Ursprungsknotens geschlossen werden. Durch
die Speicherung des Index in einer seperat von Boost bereitgestellten, mitschrumpfenden Datenstruktur kann immernoch eindeutig auf den Originalknoten geschlossen werden. 

Eine einfache Möglichkeit der Ausgabe wird ebenfalls von Boost gestellt. Boost bietet eine einzelne Funktion, welche einen Boost-Graphen im Graphviz-Format ausgiebt. Dadurch ergibt sich eine schnelle 
Möglichkeit, die Richtigkeit der Lösung zu kontrollieren. Gerade bei frühen Versuchen der Nachbarschaftssuche wurde viel mit diesem Werkzeug gearbeitet, um die Ergebnisse zu verifizieren. 

\subsubsection{Hilfsroutinen}
Neben Graphen wurden auch einige von Boosts Hilfsfunktionen benutzt. Um bei der Lösung des PCP möglichst flexibel zu sein wurden eine Vielzahl an Variablen verwendet, um bestimmte Parametern der 
Nachbarschaftssuche einfacher ändern zu können. Dazu wurde angedacht, die Terminalparameter, als jene Parameter, welche beim Aufruf des Programmes über die Kommandozeile mitgegeben werden, zu verwenden. 
In einem ersten Versuch wurde ein einfaches Stringparsing betrieben, welches sich aber schnell als fehleranfällig heraustellte, und zusätzlich noch zu unschönem Code führte. 

Genau für den Zweck der Programmargumente bietet Boost eine eigene Unterbibliothek, welche alle Aufgaben aus dem Bereich des Parsings abnimmt, und dies außerdem noch in einer eleganten Art und Weise für
den Programmierer anbietet. Durch einen Bibliotheksaufruf können einfach neue Befehle angegeben werden, und diesen Befehlen gleich ein bestimmter Datentyp zugeordnet werden. Es besteht nicht nur die 
Möglichkeit einen bestimmten Datentypen zuzuordnen, sondern ein bestimmtes Argument gleich an eine Variable zu binden, so dass diese Variable automatisch von Boost gesetzt wird. Des Weiteren kann einem
Argument automatisch ein Standardwert zugewiesen werden, welcher automatisch angenommen wird, sollte der Benutzer bei der Ausführung des Programmes dieses Argument nicht verwenden. 

Um eine möglichst genaue Zeitmessung für die statistische Auswertung der Ergebnisse zu erlangen, wurde die Zeitnehmungsbibliothek von Boost herangezogen. Boost stellt einfache Zeitnehmer bereit, um 
möglichst genaue Messergebnisse zu bekommen, welche im Millisekundenbereich liegen. Dies ist manchmal garnicht so einfach da etwa die normalen Zeitfunktionen nur auf Sekundenbasis arbeiten, und eine 
genauere Zeitmessung zumeist über die abgelaufenen CPU-Zyklen erfolgt.

\subsection{Valgrind}
Valgrind ist, wie schon DDD ein Werkzeug zur Fehlersuche in laufenden Programmen. Dabei spezialisiert Valgrind sich auf die korrekte Verwendung und Freigabe von Speicher. Mit Hilfe von Valgrind ist es
möglich Speicherlecks aufzuspüren und zu schließen. Dafür überwacht Valgrind den Programmablauf und gibt nach Beendigung des Programmes einen genaue Aufzählung an vergessenen Speicherbereichen und 
nicht sauber gelöschten Objekten. 

Zum Zweck der Informationssammlung, und zur besseren Darstellung der Ergebnisse sollte das Programm wieder mit der Debug-Option des Compilers übersetzt werden. Dadurch kann Valgrind unter anderem die Zeile
anzeigen, in dem das nicht gelöschte Objekt erzeugt wird, und Informationen geben, ab wann ein Speicherbereich verloren gegangen ist. Außerdem ist zu beachten dass durch diese Mitprotokollierung des 
Programmablaufes die Geschwindigkeit des selben stark beeinflusst wird. Daher ist es notwendig, bei Vorgängen welche innerhalb des Programmes durch eine Ausführungszeit beschränkt werden, diese Zeit anzupassen, 
um eine korrekte Ausführung des Programmes zu gewährleisten. 

Wenn nun innerhalb des Programmes ein Stück Arbeitsspeicher reserviert wird, merkt sich Valgrind diesen Vorgang und protokolliert mit, wieviele Zeiger auf diesen Speicherbereich zeigen. Sollte nun irgendwann
kein einziger Zeiger mehr auf den Speicherbereich verweisen, es wurden also quasi alle Variablen, die diese Information enthielten verworfen oder überschrieben, und der Speicherbereich wurde nicht wieder
freigegeben, so weiß Valgrind, dass dieser Speicher verloren gegangen ist. Sollte sich dieser Vorgang wiederholen entsteht ein immer größer werdender Pool aus nicht freigegebenen Speicherbereichen, welcher
nicht gelöscht wurde. Nachdem eine Low-Level-Programmiersprache wie C++ keine richtige automatisierte Speicherverwaltung wie andere Sprachen wie Java und C\# hat, muss der Programmierer selbst für die 
Freigabe von reservierten Speicherbereichen sorgen. Daher ist es essentiell für gute Programmierung, alle erzeugten Objekte nach ihrer Verwendung wieder freizugeben, um so den exzessiven Gebrauch von
Hauptspeicher zu verhindern.

Da die VNS in der Theorie eine endlose Schleife der Verbesserung darstellt, und daher auch das Programm eine prinzipiell unbegrenzte Laufzeit besitzen sollte, ist die Frage der Speicherverwaltung essentiell.
Selbst kleinste Speicherlecks könnten, über einen langen Zeitraum betrachtet, zum Absturz des Programmes führen. Daher bietet Valgrind eine einfach zu nutzende und übersichtliche Möglichkeit die Korrektheit
der Speicherverwaltung innerhalb des laufenden Programmes zu testen. 

\subsection{Graphviz}



\subsection{Ubigraph}
Bei Ubigraph handelt es sich um eine Plattform zur visuellen Darstellung von Graphen. Die Plattform besteht einerseits aus einem Server, welcher Befehle entgegennimmt und aus diesen Befehlen eine 
dreidimensionale Darstellung des Graphen ausgibt, und andererseits aus einer Vielzahl an Schnittstellen für alle nur erdenklichen Programmiersprachen. Mit Hilfe von Ubigraph ist es möglich, die
Vorgänge innerhalb der VNS, und auch beim Aufbau der Initiallösung zeitnah mitzuverfolgen, und gleichzeitig eine ansehnliche dreidimensionale Darstellung des Graphen zu erlangen.

Bei der Serversoftware handelt es sich um ein Stück proprietären Code, welcher per XML-RPCs angesprochen werden kann. Der Server nutzt OpenGL um dei gezeichneten Graphen per Grafikkarte darzustellen. Als
Schnittstelle dienen \textit{Remote Procedure Calls (RPC)} welche per \textit{Hyper Text Transfer Protocol (HTTP)} an den Server übermittelt werden. Der Name XML-RPC stammt von der Auszeichnungssprache
\textit{Extensible Markup Languag (XML)}, was soviel heißt wie Erweiterbare Auszeichnungssprache, welche als Anfrage- und Antwortsprache verwendet wird. Die Daten einer solchen Anfrage werden von der
ausgeführten Software per HTTP-POST-Methode an den Server übermittelt, welcher auch wieder eine XML-Meldung über Erfolg oder Misserfolg der Operation bekannt gibt. 

Da es sich bei XML-RPC um einen textbasierten Standard handelt, ist vor allem die Integration von Ubigraph in Skriptsprachen besonders einfach. Für viele Sprachen werden bereits Schnittstellen angeliefert,
welche eine noch einfachere Verwendung von Ubigraph ermöglichen. Der Quellcode dieser Schnittstellen ist als Open-Source-Software zur Verfügung gestellt, und kann daher ebenso eingesehen werden wie die
Spezifikation für die Server-Schnittstelle per RPC. Da für C++ selbst keine eigene Schnittstelle vorhanden war, wurde auf die Schnittstelle für C zurückgegriffen, welche sich wiederum einiger anderen Bibliotheken
aus dem XML und RPC-Umfeld bedient. Da C++ die Möglichkeit bietet, nahtlos mit C-Code umzugehen war die Integration von Ubigraph keine große Herausforderung. Die Ubigraphschnittstelle für C bietet einige
einfache Optionen zum Zeichnen von Knoten, Kanten beziehungsweise der Manipulation der Eigenschaften der selben, wie eben Farbe, Form und Beschriftung.


% TODO: wtf
\subsection{rsnyc}
