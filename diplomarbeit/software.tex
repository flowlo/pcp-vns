\section{Implementierung}
% TODO: everything 

\subsection{C++}
Als primäre Programmiersprache wurde C++ gewählt. Verschiedene Gründe sprachen für C++, unter anderem die an C angelehnte Syntax, aber
auch die gute Performance im Vergleich zu Sprachen, welche in einer Virtuellen Maschine laufen, oder gar interpretiert werden. Zu 
all diesen Gründen kam noch hinzu, dass für C++ bereits ausgezeichnete Codebibliotheken zur Verfügung stehen, welche ein schnelles
und effizientes Arbeiten ermöglichen.

Bei C++ handelt es sich um eine im Jahr 1979 von Bjanre Stroustrup entwickelte Programmiersprache, welche anfangs als \textit{C mit Klassen}
ausgelegt wurde. Über mehrere Entwicklungsschritte entwickelte sich eine Sprache, dessen Wurzeln in der Sprache C immer noch zu erkennen sind, 
dessen Möglichkeiten jene von C aber bei weitem übersteigen. 

Wie C handelt es sich bei C++ um eine statisch-typisierte Programmiersprache, das heißt der Compiler kann schon während der Übersetzung die
syntaktisch korrekte Verwendung von Datentypen sicherstellen. Dadurch können solche Überprüfungen während der Ausführung des Programmes
entfallen, was zu einer enormen Beschleunigung der Ausführung des Programmes führt. Außerdem ermöglicht es dem Entwickler Fehler mit
inkorrekt verwendeten Datentypen einfacher zu erkennen, da eine Fehlermeldung des Übersetzers eine genau Zeile angeben kann, in der 
der Fehler verursacht wird. 

Obwohl sich C++ wesentlich von C unterscheidet, ist es immer noch kompatibel mit C. Eine Stück C Code kann immer noch in C++ verwendet und
übersetzt werden. Dadurch können nicht nur Code-Bibliotheken aus dem C++-Umfeld verwendet werden, sondern auch ursprünglich für C gedachte
Bibliotheken eingebunden und verwendet werden. Da C immer noch eine der beliebtesten Programmiersprachen weltweit, und vor allem \textbf{die}
Sprache für Systemprogrammierung und performante Software ist, bedeutet diese Mitnahme der Vorteile aus der C-Welt in die Welt von C++
einen großen Vorteil bei der Programmierung.

Des Weiteren wurde C++ möglichst plattformunabhängig designt. Während manche Funktionen, welche vom Betriebssystem bereit gestellt werden, 
natürlich nicht wirklich plattformunabhängig sein können, wurde in C++ versucht, keine Funktionen oder Eigenschaften zu verwenden, welche
nur von einem bestimmten System ausgeführt werden können. Aus diesem Grund sind auch die meisten Code-Bibliotheken für C++ für die meisten
Plattformen erhältlich. 

Zusätzlich zu erwähnen ist die umfangreiche \textit{Standard Template Library (STL)}, welche es inzwischen sogar fast vollständig in den 
C++-Standard gebracht hat. Bei der STL handelt es sich um eine von den meisten Compilern bereitgestellte Code-Bibliothek für häufig 
verwendete Datenstrukturen und Hilfsroutinen, welche eine wesentliche Erleichterung für jeden C++-Programmierer darstellen. Unter anderem
in der STL enthaltene Datenstrukturen sind Listen, Stacks, aber auch Iteratoren für die meisten Datenstrukturen, häufig verwendete Algorithmen
wie Quicksort, Binarysearch. Hinzu kommen vereinheitlichte Ein- und Ausgabemöglichkeiten für verschiedenste Plattformen, sowie die 
Möglichkeit der parallelen Ausführung von Software mit Hilfe von Threads.

\subsection{Python}

\subsection{Make}

\subsection{Compiler}

\subsubsection{GCC}

\subsubsection{clang}

\subsection{DDD}

\subsection{Boost}

\subsubsection{graph}

\subsubsection{Hilfsroutinen}

\subsection{Valgrind}

\subsection{Graphviz}

\subsection{Ubigraph}

% TODO: wtf
\subsection{rsnyc}
