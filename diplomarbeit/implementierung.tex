\chapter{Implementierung}
%TODO: sec:ansatz
Für die Implementierung des in Abschnitt \ref{sec:ansatz} vorgeschlagenen Lösungsansatzes, wurde eine möglichst flexible Struktur des Programmes verwendet. Wie in Abschnitt \ref{sec:cpp} begründet, wurde
als Implementierungssprache C++ gewählt. Durch den Einsatz von C++ kann ein einfach erweiterbares Programmgerüst geschrieben werden, welches es erlaubt, schnell und ohne großen Aufwand mehrere Nachbarschaften
für die VNS zu implementieren. 

Um diese flexible Struktur zu ermöglichen wurde das Programm in vier unabhängige Bereiche aufgeteilt, welche durch eine einzelne Startroutine zusammengehalten werden. Diese Bereiche können grob als Parser, 
Konstruktionsheuristik, Variable Nachbarschaftssuche, und Nachbarschaften zusammengefasst werden. Hinzu kommt die sogenannte Solution, eine eigene Klasse um eine Lösung sowohl zu speichern, aber auch um Möglichkeiten
anzubieten, eine bestehende Lösung zu verändern. Diese einzelnen Bereiche sollen im folgenden Abschnitt ausführlich besprochen werden.

\section{Solution}
Bei Solution handelt es sich um die Basisklasse für alle weiteren Operationen. Sie speichert sowohl den Graphen, von dem bei der Lösung ausgegangen wird, als auch die Lösung selbst, mit konkret verteilten Farben
für einzelne Partitionen. Außerdem stellt Solution Methoden bereit, um Veränderungen an Lösungen vorzunehmen, wie sie etwa von den Nachbarschaften gebraucht werden, um einzelne Knoten umzufärben, oder zu ersetzen.
Außerdem besitzt Solution mehrere Methoden, einen Problemgraphen aus einer Datei auszulesen, und diesen wiederum in einem Solution-Objekt zu speichern. Diese Methoden zum Einlesen von Dateien werden genauer 
in Abschnitt \ref{sec:parser} besprochen.

\singlespacing
\begin{lstlisting}[caption={Ein Ausschnitt aus der Signatur der Solutionklasse},label=lst:solution]
class Solution {
   public:
      Solution();
      Solution(Solution *s);
      ~Solution(); 
      void requestDeepCopy();

      Graph *g;
      int numParts;
      int *partition;
      std::vector<Vertex> *partNodes;
      int colorsUsed;
      int *representatives;
      int *copyCounter;
};
\end{lstlisting}
\setstretch{1.5}

Gespeichert wird eine Referenz auf den Lösungsgraphen. Dieser Graph enthält ausschließlich jene Knoten, welche als Repräsentanten für ihre Partition
ausgewählt wurden. Der Graph muss daher verändert werden, wenn eine Partition ihren Repräsentanten wechselt. Des weiteren wird die Anzahl an Partitionen
gespeichert, so wie die Anzahl an verwendeten Farben für eine bestimmte Lösung. 

Im \texttt{partition}-Array werden für jede Partition genau eine ausgewählte Farbe gespeichert, wobei sich diese Farben natürlich von Lösung zu Lösung 
unterscheiden können, als Index dient die Partitions-ID. Das Array \texttt{representatives} bietet eine Zuordnung von Partition zu Knoten. Ein Wert 
an der Position $n$ zeigt auf jenen Knoten, der die Partition $n$ vertritt.

Einen Sonderfall stellt das Vector-Array \texttt{partNodes} dar. Es wird ausschließlich bei der eingelesenen Probleminstanz eingesetzt, und speichert
dort für jede Partition alle zugeordneten Knoten ab. Dies dient der schnellen Findung von alternativen Knoten für eine bestimmte Partition, und findet
in mehreren Nachbarschaften verwendung. 

\subsection{Graph-Datenstruktur}
Für die Speicherung von sowohl Lösungs als auch Problemgraphen dient die \texttt{adjacency\_list}-Datenstruktur, welche von Boost bereitgestellt wird. Der Einsatz von Boost wird in Abschnitt \ref{sec:boost} 
beschrieben. Es wurde bewusst eine Liste als Grundlage der Speicherung verwendet, um ein schnelles Iterieren über Knoten und Kanten zu ermöglichen. Die ebenfalls von Boost bereitgestellte 
\texttt{adjacency\_matrix} hatte sich in einem Test als ungeeignet erwiesen, da nicht alle Operationen, wie das hinzufügen und löschen von Knoten unterstützt werden. Des Weiteren wurde die Option für 
ungerichtete Grapehn gesetzt, da das PCP keine gerichteten Beziehungen zwischen Knoten vorsieht. Dies führt dazu, dass, sobald eine Kante vom Knoten \textit{A} zu Knoten \textit{B} führt, automatisch auch
eine Beziehung von Knoten \textit{B} zu Knoten \textit{A} führt.

Boost bietet außerdem die Wahl zwischen verschiedenen Datenstrukturen für die Speicherung des Graphobjektes. Selbst wenn die Auswahl auf eine Liste gefallen ist, gibt es noch mehrere Möglichkeiten die Datenstruktur
zu beeinflussen. Zum einen kann bei der Speicherung der Knoten sowohl eine normale Liste, als auch ein Vector, ausgwählt werden. Die selbe Möglichkeit präsentiert sich dann auch für die Speicherung der Kanten.
Ein Vector ist eine Datenstruktur ähnlich einem Array, bei dem alle Datenelemente im Speicher hintereinander angeordnet sind, welche aber vergrößert oder verkleinert werden kann.

Für die Speicherung der Knoten wurde ein Vector herangezogen, zum einen, da nun ein Knoten direkt über einen zahlenbasierten Index ansprechbar ist, zum Anderen, weil während der Laufzeit des Programmes nur
zwei Mal Knoten hinzugefügt oder entfernt werden. Einmal beim Einlesen der Probleminstanz aus einer Datei, und einmal bei der Berechnung der Ausgangslösung. Bei allen anderen Veränderungen der Lösung erfolgt
der Austausch von Knoten direkt an der Stelle des zu ersetzenden Knoten. Es ist daher anzunehmen das während der Laufzeit des Programmes keine größeren Reallokationsoperationen vorgenommen werden müssen.

Schwieriger ist die Wahl schon bei der Frage, ob für die Speicherung der Kanten eine Liste nicht besser geeignet wäre. Da in verschiedenen, weiter unten beschriebenen, Nachbarschaften Knoten ausgetauscht werden,
ein Vorgang der darauf basiert, die alten Adjazenzen eines Knotens zu kappen, und an deren Stelle neue Adjazenzen zu schaffen, welche dem ausgetauschten Knoten entsprechen, handelt es sich bei der Kantenspeicherung
um eine Datenstruktur die häufigen Änderungen unterliegt. Da die meisten Knoten aber eine ähnliche Anzahl an Adjazenzen haben, würde auch hier ein Vektor keine allzu häufigen Reallokationen durchführen müssen.
In einem Test ergab sich unter Verwendung eines Vectors als Speicherstruktur für die Adjazenzinformationen ein bis zu 20\% höhere Geschwindigkeit im Vergleich zur Verwendung einer Liste. Dies ist wohl vor allem
auf die höhere Datenlokalität zurückzuführen, durch die der Prozessor bereits ganze Speicherbereiche in den Cache laden kann, wenn über die Adjazenzen iteriert wird.

\subsection{Kopierkonstruktor}
\label{sec:copy}

Besonders interessant bei der Solution-Klasse ist die automatisch bereitgestellte Kopiermechanik. Einer der besonderen Eigenschaften von C++ ist es, für jede Klasse automatisch einen sogenannten Kopierkonstruktor
bereitzustellen, welcher als Parameter ein Objekt des selben Types entgegennimmt, und dieses kopiert. Da im Falle einer Lösung aber nicht immer alle Daten kopiert werden müssen, wurde in dieser Arbeit eine
eigene Kopiermechanik implementiert. Wie in Listing \ref{lst:copy} zu sehen ist, wird ein Großteil der Referenzdatentypen direkt übernommen, da diese häufig garnicht verändert werden. 

Das Kopieren einer Lösung kommt vor Allem bei den verschieden Nachbarschaften zum Einsatz. Eine bestehende Lösung soll verändert werden, ohne, dass die Information der Ausgangslösung verloren geht. Also wird
die bestehende Lösung kopiert, und die Veränderungen nur auf der neuen Lösung ausgeführt. Da viele der Nachbarschaften darauf basieren, dass ausschließlich die Lösungsfarben verändert werden, kann zum Beispiel, 
anstatt den gesamten Graphen zu kopieren, eine Referenz auf den alten Graphen aufrecht erhalten werden. Selbiges gilt natürlich auch für alle direkt mit dem Graph in Verbindung stehenden Datenspeichern. 

\singlespacing
\begin{lstlisting}[caption={Der Kopierkonstruktor der Solutionklasse},label={lst:copy}]
// Copy an existing solution, this will use the same graph object as the 
// original, to cut these ties, use requestDeepCopy()
Solution::Solution(Solution *toCopy) {
   this->partNodes = NULL;
   this->g = toCopy->g;
   this->copyCounter = toCopy->copyCounter;
   // increment copycounter, to keep track of referencing solution
   *this->copyCounter += 1;
   this->numParts = toCopy->numParts;
   this->partition = new int[this->numParts];
   this->representatives = toCopy->representatives;
   this->colorsUsed = toCopy->colorsUsed;
   this->partitionMap = get(vertex_index1_t(), *this->g);
   this->idMap = get(vertex_index2_t(), *this->g);
   
   for (int i = 0; i < this->numParts; i++) {
      this->partition[i] = toCopy->partition[i];
   }
}
\end{lstlisting}
\setstretch{1.5}

Nun könnten aber Probleme entstehen. Zum einen, wenn eine Nachbarschaft beispielsweise Knoten aus dem zu Grunde liegenden Graphen austauscht, um eine bessere Lösung zu erhalten. Da durch das unechte Kopieren
der in einem Solutionobjekt referenzierte Graph auch bei anderen Lösungen zum Einsatz kommt, würde eine Veränderung auf einem Graphen, zu Veränderungen und aller Wahrscheinlichkeit nach zu Fehlern in alternativen
Lösungen führen, welche dann unweigerlich unzulässige Ergebnisse zur Folge hätten. Daher kann mit der Methode \texttt{requestDeepCopy} eine vollständige Kopie des Solutionobjektes angefordert werden. In dieser
Methode werden dann alle verbleibenden Fremdreferenzen nocheinmal kopiert, und in eigens für das anfragende Solutionobjekt reservierte Speicherbereiche geschrieben.

\singlespacing
\begin{lstlisting}[caption={Die Methode \texttt{requestDeepCopy} of the Solution class},label={lst:deepcopy}]
// Cut ties between a copy and the underlying graph by 
// making a copy of the graph
void Solution::requestDeepCopy() {
   // Copy graph
   Graph *cp = g;
   g = new Graph(*g);
   // Reset property maps
   this->partitionMap = get(vertex_index1_t(), *g);
   this->idMap = get(vertex_index2_t(), *g);

   // Copy representatives array
   int *rep = representatives;
   representatives = new int[numParts];
   for (int i = 0; i < numParts; i++) {
      representatives[i] = rep[i];
   }

   // delete the old copy if counter reached zero
   *copyCounter -= 1;
   if (*copyCounter <= 0) {
      delete copyCounter;
      delete cp;
      delete[] rep;
   }
   copyCounter = new int;
   *copyCounter = 1;
}
\end{lstlisting}
\setstretch{1.5}

Ein anderes Problem entsteht beim Aufräumen von Objekten. Sollte durch das Löschen eines Solutionobjektes auch dessen referenzierter Graph verschwinden, würde das zu Speicherfehlern bei anderen Objekten führen, 
welche auf den selben Graphen zeigen. Daher wurde der Destruktor, welcher zum Aufräumen eines Objektes aufgerufen wird, überschrieben, und löscht nun nur noch referenzierte Objekte, wenn kein anderes Objekt
mehr Referenzen auf das selbe Objekt hält. Um dies zu ermöglichen wird von allen Solutionobjekten, welche auf den selben Graphen referenzieren ein Counter referenziert, welcher die Anzahl der
Verweise auf den selben Graphen mitzählt. So lange dieser Zähler über eins ist, wird kein Objekt gelöscht. Wird nun der Destruktor eines Solutionobjektes aufgerufen, überprüft dieser, ob er die letzte
Referenz auf diese Objekte hält. Wenn ja, löscht er diese und seine eigenen Referenzen, wenn nicht, löscht er wiederum nur seine eigenen Referenzen und vermindert den Zähler um eins.

\begin{lstlisting}[caption={Der Destruktor der Solutionklasse mit Rücksichtnahme auf eventuelle verbleibende Referenzen},label={lst:destruct}]
// Destructor
Solution::~Solution() {
   // Decrement copyCounter, if <= 0 delete all information about the referenced
   // graph and corresponding arrays
   *copyCounter -= 1;
   if (*copyCounter <= 0) {
      delete g;
      delete copyCounter;
      delete[] representatives;
   }
   delete[] partition;
   delete[] partNodes;
}
\end{lstlisting}

\subsection{StoredSolution}
Um eine einzelne Lösung von einer anderen zu unterscheiden, sind zwei Informationen notwendig: die Auswahl der Repräsentanten, und die Auswahl 
an Farben. Mit diesen beiden Informationen, und dem Ausgangsgraphen, aus dem die Lösung berechnet wurde, ist eine Lösung eindeutig rekonstruierbar.
Umzu verhindern, dass die VNS-Schleife immer wieder auf die selben Lösungen kommt, ohne echte Verbesserungen zu erzielen, werden alle neuen, verbesserten
Lösungen komprimiert gespeichert. 

Zur komprimierten Speicherung wurde eine eigene Datenstruktur definiert, welche eben nur jene oben erwähnten Informationen abspeichert. 
Dazu werden zwei Arrays verwendet, welche einerseits die für einzelne Partitionen verwendeten Farben, andererseits aber auch die 
eindeutige Identifikationsnummern der Repräsentanten abspeichern. Um eine Solution einfach in solch eine StoredSolution umzuwandeln, wurde
ein Konstruktor definiert, welcher ein klassisches Solutionobjekt übernimmt. 

\singlespacing
\begin{lstlisting}[caption={Die Signatur von StoredSolution},label={lst:stored}]
struct StoredSolution {
   StoredSolution(Solution& toStore);
   ~StoredSolution();
   
   int n;
   
   int *colors;
   int *representatives;
   
   std::string toString();
};
\end{lstlisting}
\setstretch{1.5}

Für den schnelle Zuordnung einer neuen Lösung zu den bestehenden Lösungen wird eine Hashtable-Datenstruktur verwendet. Durch die Verwendung einer
Hashtable ist eine quasi sofortige Zuordnung einer Lösung zu dem entsprechenden Platz in der Hashtable möglich. Sollte eine Lösung schon einmal 
aufgekommen sein, wird dieser Lösungsstrang verworfen, da er nicht zu neuen Lösungen führen würde. Die Implementierung der Hashtable wird von 
Boost, beschrieben im Abschnitt \ref{sec:boost}, bereitgestellt. Um die Hashtable zu betreiben sind außerdem noch eine Methode zur Hashberechnung,
sowie eine Methode zum Vergleich zweier gespeicherten Solutions notwendig. Der Hash für ein StoredSolutionobjekt wird aus den gesammelten Werten
der beiden internen Arrays berechnet, während für die Vergleichsmethode die beiden Arrays Wert für Wert verglichen werden. Die von Boost 
bereitgestellte Hashtable verwendet bei gleichen Hashwerten aber unterschiedlichen Lösungen eine einfache Liste, um die Ergebnisse zu speichern.

\subsection{Plausibilitätsprüfung}

Um eine manuelle Überprüfung der Korrektheit einer Lösung zu umgehen, wird innerhalb der VNS, siehe Abschnitt~\ref{sec:impl:vns}, eine Plausibilitätsprüfung durchgeführt. Diese Überprüfung stellt an Hand 
der innerhalb einer Solution gespeicherten Eigenschaften, und an Hand des ursprünglichen Problemgraphen fest, ob es sich um eine valide Lösung handelt oder nicht. Diese Überprüfung erfolgt standardmäßig
am Ende der VNS, und überprüft das beste erzielte Ergebnis, welches dann auch als Endergebnis präsentiert wird. Mit Hilfe eines Programmparameters, welcher zu Ausführungsstart angegeben werden muss, ist es
auch möglich eine solche Überprüfung für jede Lösung einzuschalten, welche eine Verbesserung zur vorherigen darstellt. 

Innerhalb dieses Checks werden verschiedenste Dinge überprüft. Zum einen wird getestet, ob zwei verbundenen Knoten die selbe Farbe zugeteilt wurde. Dann wird überprüft, ob die gespeicherte Anzahl an Farben
mit der tatsächlich verwendeten Farben übereinstimmt. Dies ist besonders wichtig, da ja die Anzahl an Farben das Hauptkriterium für eine bessere Lösung darstellt. Dann wird überprüft, ob alle Partition
vertreten sind, und ob die jeweiligen Repräsentanten richtig abgespeichert wurden. 

Zu letzt wird noch überprüft, ob der Lösungsgraph mit dem Ursprungsgraphen übereinstimmt. Dazu werden verschiedene Fakten überprüft. Zum einen werden fehlende oder überschüssige Kanten in der Lösung gesucht, 
zum anderen wird überprüft, ob die ausgewählten Repräsentanten einer Partition wirklich zu dieser Partition zugehörig sind. 

Sollte keine dieser Überprüfungen auf Fehler stoßen, wird angenommen, es handle sich um eine valide Lösung für das PCP. Durch diese Überprüfung konnten auch schon Fehler innerhalb einzelne Nachbarschaften 
nachgewiesen und schließlich behoben werden.


\section{Parser}
\label{sec:parser}

Damit erzielte Ergebnisse mit älteren Programmversionen, aber auch mit anderen Lösungs\-ansätzen für das PCP verglichen werden können, ist es von
Vorteil, immer wieder die selbe Eingabe zu verwenden. Außerdem bietet sich eine solche Möglichkeit an, wenn, wie in dieser Arbeit ein Zufallsfaktor
zu einer Verbesserung der Lösung führen kann. Durch immer wieder gleiche Eingaben kann ein direkter Vergleich auch zwischen unterschiedlich
parametrisierten Programminstanzen angestellt werden.

Es gibt mehrere Dateiformate, mit denen Probleminstanzen des PCP abgespeichert werden können. In dieser Arbeit wurde vor Allem auf jenes
Format Wert gelegt, in dem auch die meisten Testinstanzen der einschlägigen Literatur vorliegen. Allerdings wurde im Zuge von ausgeweiteten 
Testungen auch ein zweites Format zum einlesen eines Problemgraphen implementiert. 

Eine \texttt{.pcp}-Datei ist eine einfache Textdatei, welche allerdings nach einer bestimmten Struktur aufgebaut ist. Die aller erste Zeile
einer solchen Datei besteht aus 3 Zahlen, welche jeweils durch ein Leerzeichen getrennt sind. Diese 3 Zahlen stehen für die Anzahl der Knoten, 
die Anzahl der Kanten, und die Anzahl der Partition im Problemgraph, genau in jener Reihenfolge. Nun folgen \textit{n} Zeilen an Zahlen, mit 
genau einer Zahl pro Zeile, wobei \textit{n} die Anzahl an Knoten ist. Diese Zahlen stehen für die Partitionen, denen die Knoten zugeordnet werden.
Die erste Zeile beinhaltet also die Partition des ersten Knotens, die zweite Zeile die des zweiten Knotens und so weiter. Zu letzt folgen noch
\textit{m} Zeilen mit jeweils einem Zahlenpaar, welche die Kanten zwischen den Knoten definieren, wobei \textit{m} der Anzahl an Kanten insgesamt
entspricht.

\singlespacing
\begin{lstlisting}[caption={Eine einfache \textit{.pcp}-Beispieldatei},label={lst:pcp}]
4 5 2
0
1
1
0
0 1
0 2
1 2
1 3
2 3
\end{lstlisting}
\setstretch{1.5}

Der Parser, welcher eine solche Datei einliest, ist in die Klasse Solution eingearbeitet. Die statische Methode \texttt{fromPcpStream} ließt eine
Datei von einem Eingabestream ein, und wandelt sie in ein Solutionobjekt mit entsprechendem Problemgraphen um. Dieses Solutiobjekt wird während
der gesamten Laufzeit des Programmes nicht verändert oder gelöscht, und dient als Ausgangspunkt der Konstruktionsheuristik, aber auch als
Referenz für Nachbarschaften, welche einzelne Knoten austauschen. Außerdem speichert diese Solutionobjekt als einziges eine direkte Zuordnung von einer Partitions-ID
zu den verfügbaren Knoten in der Partition ab. 

\section{Konstruktionsheuristik}
Als Konstruktionsheuristik wurde der in Abschnitt \ref{sec:onestep} Algorithmus OneStep-CD gewählt. Die verwendete Implementierung des Algorithmus
erzeugt ein neues Solutionobjekt, als einziger Übergabeparameter ist das aus der Datei eingelesen Solutionobjekt, welches ja den vollständigen Problemgraphen
enthält. Dieses Solutionobjekt bildet dann die Ausgangsbasis für die Verbesserung in der Variablen Nachbarschaftssuche. Der Graph, welcher in dem erzeugten Solutionobjekt
referenziert wird, ist ein anderer Graph als in der Gesamtlösung verwendet wird. In dem referenzierten Graphen befinden sich nur mehr ein Knoten per Partition, welcher auch den
ausgewählten Repräsentanten der jeweiligen Partition darstellt. 

\subsection{Pilot}

%TODO: pilot, HERBERT TRINK DAS!

\section{Nachbarschaftssuche}
\label{sec:impl:vns}

\begin{algorithm}
\begin{algorithmic}[1]
\State Berechne Initiallösung mit \emph{onestepCD}
\While{Terminationsbedingung \textbf{nicht} erfüllt}
\State Führe Variable Neighborhood Descent mit Best Improvement Strategie durch:
\State Nachbarschaft $l \leftarrow 1$
\While{$l \leq 3$ \textbf{und} Zeitlimit nicht erreicht}
\State Führe Nachbarschaft $n_l$ aus
\If {Lösung verbessert \textbf{und} $l\neq 1$} 
\State  $l\leftarrow 1$  
\Else
\State  $l\leftarrow l + 1$
\EndIf
\EndWhile
\If{Beste Lösung verbessert \textbf{oder} $k \geq k_{\mathrm max}$}
\State $k \leftarrow k_{\mathrm start}$ 
\Else
\State $k \leftarrow k + 1$
\EndIf
\State Wähle zufällig Nachbarschaft aus und ``schüttle" Lösung mit $k$ zufälligen Zügen
\EndWhile
\State\Return Beste gefundene Lösung
\end{algorithmic}
\caption{Pseudocode der Variablen Nachbarschaftssuche}
\end{algorithm}

\section{Nachbarschaften}
Die Nachbarschaften bilden das Herzstück der VNS. Eine einzelne Nachbarschaft versucht kleine Veränderungen vorzunehmen, welche die Lösung verbessern, und gibt diese Verbesserungen dann an weitere Nachbarschaften weiter.
Jede Nachbarschaft verfolgt ihren eigenen Ansatz zur Verbesserung, und muss nicht immer bessere Ergebnisse liefern. 

Bei der Implementierung der Nachbarschaften wurde versucht, ein möglichst einfach erweiterbares Gerüst zu schaffen, um schnell und einfach neue Ideen für Nachbarschaften ausprobieren zu können. Jede Nachbarschaft
wurde daher als eigene Klasse implementiert, welche aber alle von einer einzelnen Basisklasse erben. Diese Basisklasse definiert ein Grundgerüst, welches von jeder Nachbarschaft ausgefüllt werden muss. 

Zu diesem Grundgerüst zählt nicht nur die Implementierung der eigentlichen Nachbarschaft, sondern auch eine Funktion um den Namen der Nachbarschaft auszugeben, ein Funktion um ein Kurzzeichen für die Nachbarschaft
zu definieren, und eine Funktion um eine bestehende Lösung zu schütteln, wie in Abschnitt~\ref{sec:vns} beschrieben. Jede dieser Funktionen muss für eine Nachbarschaft implementiert werden, und jede Nachbarschaft
sollte nur gültige Lösungen zurückliefern, oder zumindet einen Indikator liefern, dass die zurückgelieferte Lösung ein unbrauchbares Ergebnis darstellt.

\subsection{changeColor}
\subsection{changeNode}
\subsection{DSATUR}
\subsection{changeAll}


 
