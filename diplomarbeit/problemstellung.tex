\section{Problemstellung}
In der heutigen Zeit, in der das weltweite Kommunikationsbedürfnis in ungeahnte Höhen steigt, und, bis auf weiteres, kein Ende des Anstiegs in Sicht ist, werden
Computer-Netzwerke, besonders jene, welche ganze Kontinente umspannen, immer stärker beansprucht. Dieser Anstieg ist vor allem mit dem gesteigerten, und sich
grundlegend ändernden Konsumverhalten der Menschen in der heutigen Zeit zurückzuführen. Immer mehr Informationen in Netzwerken werden nicht in Form von Texten
sondern, aufwendig aufbereitet, als Bild und Ton zum Endkunden übertragen. Dieses Bild- und Tonmaterial erzeugt natürlich eine höhere Auslastung von Datenleitungen,
was, im Zusammenspiel mit der gestiegenen Anzahl an Informationsbedürftigen insgesamt, zu einer Verknappung der Ressourcen in Computer-Netzen führt.

Um diesen neuen Gegebenheiten gerecht zu werden, gibt es zwei Möglichkeiten: Entweder ein Ausbau, und Neubau von Netzen, oder eine bessere Ausnutzung von bestehenden
Netzen. Ausbau und Neubau bringen allerdings neue Probleme mit sich. Erstens ist Netzwerkinfrastruktur teuer, besonders wenn sie über mehrere Jahrzehnte hinweg
halten und ihren Dienst verrichten soll, ohne eine ständige Wartung mit sich zu ziehen. Zweitens ist es manchmal aus verschiedenen Gründen gar nicht möglich, 
ein Netz weiter auszubauen. Sollte sich eine bestehende Infrastruktur bereits in ihrer letzen Ausbaustufe befinden, ist es überhaupt notwendig ein gänzlich
neues Netz aufzubauen, was wiederum mit noch höheren Kosten verbunden ist.

Die zweite Möglichkeit, die bessere Ausnutzung von bestehenden Netzen, kann zwar die letztendliche Notwendigkeit von neuen Netzen nicht verhindern, den Zeitpunkt
ihres notwendigwerdens aber hinauszögern. Durch eine optimierte Nutzung können bereits vorhandene Netze über einen längeren Zeitraum genutzt werden, ohne einen
Engpass zu bilden. Wie diese Optimierungen im konkreten Fall aussehen hängt unter anderem von der Größe und der Art des Netzes, aber auch von der Art des
Kommunikationsbedürfnisses ab. In dieser Arbeit konzentrieren wir uns auf rein optische Netze mit einer ganz genau definierten Art des Kommunikationsbedürfnisses.

In rein optischen Netzen, also Netzwerken, in denen Information rein über optische Leiter wie Glasfaser übertragen wird und das Signal auf der ganzen Strecke zu 
keinem Zeitpunkt in ein elektrisches Signal zurückgewandelt wird, kann \textit{Wavelength Division Multiplexing}
(WDM) zur parallelen Übertragung mehrer Datenströme über ein und die selbe Leitung betrieben werden. Die Idee hinter WDM ist es, für zwei verschiedene Datenströme 
unterschiedliche Farben von Licht zu verwenden. Der englische Name Wavelength Division (deutsch: Wellenlängen Unterteilung) kommt von der Tatsache, dass 
verschiedene Farben von Licht verschiedene, eindeutige Wellenlängen haben. Da sich die unterschiedliche Farben des Lichtes nicht gegenseitig beeinflussen, können
viele Farben gleichzeitig durch ein und den selben Lichtwellenleiter geschickt werden, und am anderen Ende des Leiters wieder problemlos auseinander gehalten werden.
Dies funktioniert sogar so gut, dass Lichtwellen, welche sich nur um wenige Nanometer an Wellenlänge unterscheiden, immer noch eindeutig zuordenbar sind.

Des Weiteren wird oft angestrebt, einem einzelnen Datenstrom auf seinem gesamten Weg durch das Netzwerk ein und die selbe Farbe zu geben. Dies bietet Vorteil vor
Allem im Bereich der Einfachheit der Implementierung, da optische Switche innerhalb des Netzwerkes keine Umwandlung von einer Farbe in eine andere vornehmen müssen.
Hinzu kommt ein geringerer Stromverbrauch im Netzwerk, da ein Umwandlung von einer Wellenlänge in eine andere auch ein gewisses Maß an zusätzlichem Strom benötigen
würde.

Zu guter Letzt ist noch die spezielle Art des Kommunikationsbedürnisses zu beachten. Anders als zum Beispiel im Internet, wo die Kommunikation für einen Netzbetreiber
unberechen- und vor Allem unverhersehbar abläuft, gehen wir hier von einem Fall aus, in dem alle Kommunikationswünsche bereits von vorn herein bekannt sind. Vorstellbar
wäre solch ein Fall zum Beispiel bei einer gemieteten Standleitung eines Unternehmens zu ihrer Zweigstelle am anderen Ende des Landes. Auch wenn dies nun nach einer
starken Einschränkung des Einsatzgebietes aussieht, ist der Nutzen dennoch nicht zu unterschätzen. Wenn zum Beispiel der Aufwand für den Betrieb solcher Standleitungen
minimiert werden kann, könnten mehr Ressourcen auf die Erfüllung der vorhin erwähnten unvorhersehbaren Kommunikationsbedürfnisse investiert werden.

\subsection{Partition Graph Coloring Problem}
Mit Hilfe des Partition Graph Coloring Problem (PCP) wird versucht, diesen sehr spezifischen Anwendungsfall mathematisch zu beschreiben. Um das Problem zu lösen wird
ein Graph aufgebaut, mit dessen Hilfe dann eine optimierte Lösung für das Problem berechnet werden kann. Ein Graph ist ein mathemaitsches Konstrukt, dass aus Knoten
besteht, die durch Kanten verbunden werden. Ein Knoten A kann mit Knoten B durch so eine Kante verbunden werden, Knoten A muss aber zum Beispiel keine Verbindung mit
Knoten C besitzen. Wie gezeigt wurde, handelt es sich bei dem PCP um ein NP-hartes Problem, das heißt es kann nicht in polynomieller Laufzeit auf einem Computer
wie man ihn heute kennt eine Lösung gefunden werden, welche mit 100\%-iger Sicherheit die optimale Lösung für das konkrete Problem darstellt. Polynomielle Laufzeit
wird im Normalfall als Grenze zwischen Problemen angesehen, welche in vertretbarar Laufzeit noch zu berechnen sind, und Problemen für die der Lösungsaufwand in
keinem Verhältnis zum Ergebnis steht.

\begin{figure}
	\centering
	\begin{subfigure}{\textwidth}
		\includegraphics{img/bsp1}
		\caption{Repräsentatives Beispiel für ein rein optisches Netzwerk innerhalb 
		Österreichs. Es gibt 2 Kommunikationswünsche, zwischen A und B sowie zwischen C und D.}
		\label{fig:example:a}
	\end{subfigure}
	\begin{subfigure}{\textwidth}
		\includegraphics{img/bsp2}
		\caption{Beispiel für verschiedene Routen zwischen insgesamt 4 Kommunikationspartnern.
		Verschiedene Möglichkeiten der Wegfindung zwischen A und B werden in Hell- und Dunkelblau sowie Violett dargestellt. Die alternativen Routen zwischen C und D sind
		Gelb und Orange eingefärbt.}
		\label{fig:example:b}
	\end{subfigure}
	\caption{Ein Beispiel für eine Probleminstanz des Partition Graph Coloring Problems}
	\label{fix:example}
\end{figure}

\begin{figure}
	\centering
	\begin{subfigure}[t]{0.3\textwidth}
		\includegraphics[width=\textwidth]{img/bsp3}
		\caption{Der aus Abbildung \ref{fig:example:b} generierte Problemgraph. Jede Route wird als Knoten repräsentiert. Die verwendete Farbe stimmt mit den
		für die Routen verwendeten Farben überein. Die Partitionierung wird durch das angedeutete Oval im Hintergrund der Knoten angedeutet.}
		\label{fig:example:c}
	\end{subfigure}
	\begin{subfigure}[t]{0.3\textwidth}
		\includegraphics[width=\textwidth]{img/bsp4}
		\caption{Der Problemgraph wird um die Beziehungen der Knoten untereinander erweitert. Jede Route, welche eine Teilstrecke mit einer anderen Route
		gemeinsam hat, wird mit dieser Route verbunden. Die Kanten zwischen den Knoten stellen diese Beziehung dar.}
		\label{fig:example:d}
	\end{subfigure}
	\begin{subfigure}[t]{0.3\textwidth}
		\includegraphics[width=\textwidth]{img/bsp5}
		\caption{Eine mögliche Lösung des PCP. Von jeder Partition muss nur ein Knoten ausgewählt werden, alle anderen werden ausgeblendet. Die Auswahl
		für die Strecke zwischen A und B hat kein gemeinsames Teilstück mit der für die Strecke C-D getroffene Auswahl. Daher kann ein und die selbe Farbe
		für beide Übertragungen verwendet werden.}
		\label{fig:example:e}
	\end{subfigure}
	
	\caption{Aufbau und Lösung eines Problem-Graphen des Partition Graph Coloring Problems}
\end{figure}

Um das Problem besser lösen zu können, wird ein einheitliches Ziel, und eine einheitliche Eingabe benötigt.
Dazu wird die Problemstellung nun in einem ersten Schritt auf folgende Punkte fixiert:
\begin{itemize}
	\item Alle Kommunikationsbedürfnisse, welche aus einem Start- und einem Endpunkt bestehen, sind bekannt.
	\item Für jedes Kommunikationsbedürfnis gibt es einen oderer mehrere Wege, sprich Routen, zwischen Start- und Endpunkt.
	\item Sollte eine Route einen Teil des Weges auf der selben Strecke zurücklegen wie eine andere Route, müssen sie unterschiedliche Farben verwenden.
	\item Jede Route verwendet genau eine Farbe, und zwar für die gesamte Strecke.
	\item Die Lösung besteht aus genau einer Route für jedes Kommunikationspaar, und genau einer dazugehörigen Farbe.
\end{itemize}

Aus diesen Bedingungen kann ein Graph abgeleitet werden, der genau jene Bedürfnisse erfüllt, um eine einfaches lösen des Problems zu ermöglichen. Der Problemgraph
aus dem dann die Lösung berechnet wird, besteht aus Partitionen, welche eine Menge von Knoten abbilden und Kanten, welche die einzelnen Knoten verbinden.
Ein Kommunikationsbedürfnis zwischen zwei Partnern wird als eine Menge von Knoten gesehen. Die Knoten repräsentieren je eine Route zwischen den beiden Partnern. Ist also nur eine Route zur Auswahl
zwischen zwei Kommunikationsendpunkten, gibt es nur einen Knoten in dieser Menge. Diese Menge an Knoten wird Partition genannt, und ist damit namensgebend für das PCP. Dieser
Schritt wird in Abbildung \ref{fig:example:c} veranschaulicht.

Die Kanten bilden nun die Beziehungen zwischen verschiedenen Routen ab. Teilen sich zwei Routen das selbe Teilstück im optischen Netzwerk, werden sie mit einer Kante verbunden.
Zu beachten ist, dass dies nicht nur für Knoten in der selben Partition gilt, sondern auch für Knoten aus unterschiedlichen Partitionen. Verwendet also Route 1 der
Kommunikation $(AB)$ ein Stück der Strecke, das auch von der Route 2 der Kommunikation $(CD)$ verwendet wird, werden diese beiden Routen, welche ja jeweils als Knoten
repräsentiert werden, im Ausgangsgraphen mit Hilfe einer Kante verbunden. Außerdem ist noch zu beachten, dass für jedes in dieser Art verbundene Routenpaar nur eine Kante
verwendet wird. Teilen sich Routen 1 und 2 mehr als ein Teilstück, so werden sie trotzdem nur einmal per Kante verbunden. Die Vervollständigung des Problemgraphen
wird in Abbildung \ref{fig:example:d} dargestellt.

Nachdem der Ausgangsgraph generiert wurde, kann eine Lösung für das PCP berechnet werden. Dazu muss für jede Partition genau ein Knoten ausgewählt werden, und diese
Knoten dann so eingefärbt werden, dass sich keine 2 mit einer Kante verbunden Knoten die gleiche Farbe teilen. Der letztere Teil, also das einfärben von Knoten in
Abhängigkeit der Farben der Nachbarknoten ist als klassisches Graph Coloring bekannt. Es kommt zum Beispiel bei der Einfärbung von politischen Landkarten zum Einsatz, 
wo benachbarte Länder nicht die selbe Farbe haben dürfen. Was das PCP unterscheidet, ist die Möglichkeit, nur einen Knoten aus einer Partition als Repräsentanten
auszuwählen, und die anderen quasi auszublenden. Ein Beispiel einer solchen Lösung des Problems ist in Abbildung \ref{fig:example:e} zu finden.

Die Schwierigkeit des Problems liegt in dem Zusammenspiel von verschiedenen Möglichkeiten, ein und den selben Graphen einzufärben, und andererseits mit einer anderen 
Auswahl an Knoten gleich den gesamten Lösungsgraphen zu verändern. Das Kriterium für die Güte einer Lösung ist die Anzahl der verwendeten Farben. Um sich nur auf diese 
Anzahl konzentrieren zu können, wird angenommen, dass jede Route zwischen zwei Kommunikationspartner gleich gut ist. Eventuelle Umwege die durch eine längere Route als nötig 
entstehen werden also im PCP ignoriert. 

Außerdem geht man bei der Lösung des PCP von einem homogenen Netzwerk aus, in dem jede Leitung jede Wellenlänge unterstützt. In der Realität kann es durchaus vorkommen, dass ein
Lichtwellenleiter, welcher aus einem chemisch minimal anders zusammengesetzten Stoff besteht als seine Gegenstücke in anderen Teilen des Netzwerkes, für bestimmte Wellenlängen
ungünstige Eigenschaften hat. 

\subsection{Ausgangsmaterial}
Das Partition Graph Coloring Problem beschränkt sich nur auf den Weg vom Problemgraphen zum Lösungsgraphen. Für die Findung alternativer Routen durch ein Netzwerk gibt es 
bereits ausgiebig beschriebene und getestete Algorithmen, welche auch auf einem optischen Netzwerk anwendbar sind. Mit Hilfe dieser Algorithmen können die für den
Problemgraphen benötigten Knoten für jedes Kommunikationspaar gefunden werden. Wichtig hierbei ist es, dass es wenn möglich mehrere Alternativen für eine Übertragung geben sollte,
da es sich sonst um ein normales Graph Coloring Problem handeln würde. Die Findung der Kanten, welche Knoten verbinden, die die selbe Teilstrecke verwenden, ist auch ein
gelöstes Problem, welches das PCP nicht beschäftigt.

Das PCP generiert eine Lösung aus dem Problemgraphen, welche eindeutig für jeden Kommunikationswunsch eine Route und eine Farbe zuordnet. Je dichter der Problemgraph, dass heißt
je mehr Kanten die verschiedenen Knoten verbinden, desto schwieriger ist es eine geringe Anzahl an Farben für die optimierte Lösung zu finden. Dabei legt ein Algorithmus, der das
PCP löst, nicht strikt die zu verwendenden Wellenlängen fest, sondern weißt gleichen Wellenlängen nur die gleiche Zahl zu. Gleiches gilt für die Auswahl einer Route. Anstelle
einer fixen Beschreibung, welchen Weg die ausgewählte Route durch das Netzwerk nimmt, zeigt ein Algorithmus für das PCP lediglich auf, welcher Knoten verwendet wurde. 
Die Zuordnung von Knoten zur Route muss daher, wie auch schon bei den Wellenlängen, später erfolgen.

