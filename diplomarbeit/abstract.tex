\phantomsection
\addcontentsline{toc}{chapter}{Abstract}

\selectlanguage{english}
\begin{abstract}
The internet as global communication platform evolved to a pillar of modern society. Increasing demand to exchange data keeps putting a fundamental question to internet service providers: How to fulfill the consumers requests? When expanding communication networks one can either build new and expand exissting networks, which means being forced to install additional cables and expensive hardware, or adapt the network to utilize existing devices and lines more efficiently.

This is where the aspect of optimizing networks comes to mind. In this work we will show in detail how to model and solve the real world problem of assigning wavelengths to communication lines in fibre networks relying on Wavelength Division Multiplexing (WDM) using the Partition Graph Coloring Problem (PCP). We will present a solution using metaheuristic approaches, namely a Variable Neighborhood Search (VNS).
\end{abstract}

\phantomsection
\addcontentsline{toc}{chapter}{Zusammenfassung}

\selectlanguage{ngerman}
\begin{abstract}
Das Internet, in seiner Funktion als globale Kommunikationsplattform, entwickelte sich zu einem Stützpfeiler der modernen Gesellschaft. Das ständig steigendene Datenaufkommen stellt Internet Service Provider immer wieder vor die Frage: Wie kann man die Kundenwünsche erfüllen? Um ein Netzwerk leistungsfähiger zu machen, kann man entweder bestehende Infrastruktur ausbauen und erweitern, was natürlich auch neue, kostspielige Anschaffungen erfordert, oder das bestehende Netz so anpassen, dass es bestehende Ressourcen effizienter nützt.

Effizientere Ressourcennutzung ist eng verbunden mit Optimierung. In dieser Arbeit werden wir im Detail zeigen, wie man das Problem der Wellenlängenzuweisung in einem Glasfasernetzwerk durch das Partition Graph Coloring Problem darstellen und lösen kann. Wir werden eine ein neuartiges Lösungsverfahren für dieses Problem präsentieren, welches eine Variable Nachbarschaftssuche als Ausgangspunkt nimmt.
\end{abstract}

