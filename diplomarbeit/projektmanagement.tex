\section{Projektmanagement}

Im folgenden soll erläutert werden, wie diese Arbeit als Projekt funktioniert hat. Dabei wird im speziellen auf die Implementierung der beschriebenen Lösungsansätze und Algorithmen eingegangen, da genau dieser Aspekt wohl klar die meiste Arbeitszeit in Anspruch nahm. 

\subsection{Versionskontrolle}
Eine für das Projektteam derart wichtige Arbeit 
Eine der ersten Entscheidungen im Laufe der Arbeit am Projekt, wahrscheinlich sogar schon vor der Wahl des Arbeitstitels, war die Auswahl eines geeigneten Versionskontrollsystems. Wie bei kleineren Projekten üblich, begann die Entwicklung in einem vom Clouddienst Dropbox bereitgestellten, synchronisierten Ordner. Doch schon nach wenigen Tagen war klar, dass eine für das Projektteam derart wichtige Arbeit ein voll ausgebautes und dezidiertes Versionskontrollsystem benötigt. Schnell fiel die Entscheidung auf \textit{Git}\footnote{\url{https://git-scm.org/}}, ein System, dass mitunter von den Entwicklern der größten Projekte der \textit{Free and/or Open Source Software} eingesetzt wird.

Dabei beeindruckt Git vorallem durch den geringen Mehraufwand um die Versionskontrolle zu pflegen. Weiters wurde es so möglich komplett unabhängig an Dateien zu arbeiten und Änderungen im Nachhinein einfach zusammenzuführen.
