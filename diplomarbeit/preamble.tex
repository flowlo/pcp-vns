\usepackage[utf8]{inputenc}
\usepackage[ngerman, english]{babel}
\selectlanguage{ngerman}
\usepackage[T1]{fontenc}
\usepackage{lmodern}
\usepackage[babel, german=quotes]{csquotes}
%\usepackage{hyperref}
\usepackage[style=authoryear, natbib, backend=biber]{biblatex}
\usepackage[top=3cm,bottom=3cm,left=2cm,right=2cm]{geometry}
\usepackage{listings}
\usepackage{hyperref}
\usepackage{graphicx, fancyhdr, array, wrapfig, colortbl, algorithm, algpseudocode, subcaption, setspace}
\usepackage[usenames,dvipsnames]{xcolor}
\floatname{algorithm}{Algorithmus}
\renewcommand{\listalgorithmname}{Algorithmenverzeichnis}
\renewcommand{\algorithmicend}{\textbf{Ende}}
\renewcommand{\algorithmicif}{\textbf{Falls}}
\renewcommand{\algorithmicthen}{\textbf{dann}}
\renewcommand{\algorithmicelse}{\textbf{Sonst}}
\renewcommand{\algorithmicdo}{\textbf{}}
\renewcommand{\algorithmicwhile}{\textbf{Solange}}
\renewcommand{\algorithmicreturn}{\textbf{Returniere}}
\renewcommand{\algorithmicrequire}{\textbf{Eingabe}}
\renewcommand{\algorithmicensure}{\textbf{Ausgabe}}
\renewcommand{\algorithmicforall}{\textbf{Für alle}}
\renewcommand{\algorithmicfor}{\textbf{Für}}



\newcolumntype{L}[1]{>{\raggedright\let\newline\\\arraybackslash\hspace{0pt}}m{#1}}
\newcolumntype{C}[1]{>{\centering\let\newline\\\arraybackslash\hspace{0pt}}m{#1}}
\newcolumntype{R}[1]{>{\raggedleft\let\newline\\\arraybackslash\hspace{0pt}}m{#1}}


\lstdefinestyle{customc}{
  belowcaptionskip=1\baselineskip,
  breaklines=true,
  xleftmargin=\parindent,
  numbers=left,
  language=C++,
  showstringspaces=false,
  basicstyle=\footnotesize\ttfamily,
  keywordstyle=\bfseries\color{OliveGreen},
  commentstyle=\itshape\color{purple},
  identifierstyle=\color{blue},
  stringstyle=\color{orange},
  frame=tb,
  stepnumber=1,
  tabsize=4,
  postbreak=\raisebox{0ex}[0ex][0ex]{\ensuremath{\hookrightarrow\ \ }},
  numberstyle=\scriptsize,
  breakatwhitespace=true,
  breaklines=true,
}
\lstset{style=customc}
%\renewcommand\familydefault{\sfdefault}
\linespread{1.5}
\setlength{\parindent}{4mm}
\setlength{\parskip}{2mm}
\pagestyle{fancy}
\fancyhf{}
\clubpenalty = 300
\widowpenalty = 300
\displaywidowpenalty = 300
\fancyfoot{}
\fancyhead{}
\renewcommand{\footrulewidth}{0.4pt}
\rfoot{{\footnotesize \textnormal{\thepage}}}
\lhead{{\footnotesize \textnormal{\thesection}}}
\chead{{\footnotesize \textnormal{Variable Neighborhood Search für das Partition Graph Coloring Problem}}}
\cfoot{{\footnotesize \textnormal{Lorenz Leutgeb, Moritz Wanzenböck}}}
\lfoot{{\footnotesize \textnormal{\the\year}}}
\addbibresource{../pcp.bib}
\let\origappendix\appendix
\let\origthesection\thesection
\renewcommand\appendix{\clearpage\pagenumbering{roman}\origappendix}
\renewcommand{\lstlistlistingname}{Codeverzeichnis}
% Graphiken in ../img/
% \cite{Lu2010, Palladino2011, Noronha2006, Li2000, Frota2010, Demange, pcp-instance-page}

\title{Variable Neighborhood Search für das Partition Graph Coloring Problem}
\author{Lorenz Leutgeb, Moritz Wanzenböck}
