\documentclass[paper = a4, fontsize = 10pt]{scrartcl}

%The written project report has five parts:
%	A typewritten presentation (or essay). Hand written presentations are not admissible. The presentation should describe the project. It may be accompanied by original illustrations (graphs, drawings and photographs).It may consist of up to a maximum of 10 pages of written text (A4 format; single sided; double spaced and unbound in a minimum character size of 10 point); It may be accompanied by up to a further 10 pages of illustrations (A4 format; single sided and unbound); No extra materials such as video tapes and diskettes can be accepted as part of the typewritten presentation
%	A one page scientific summary in English containing the most important points of the project (aim of project, materials and methods, observations and conclusions).
%	A clear, concise project title in English for the Contest Catalogue. This may be accompanied, if required, by the full scientific title.
%	The full original scientific title, in the original language.
%	A straightforward description of the project of not more than ten lines in simple English for publication in the Contest Catalogue. Contestants, through their National Organiser, must ensure that this brief project description is readily understandable by the reporting press, other media, and interested members of the wider public.

\usepackage[utf8]{inputenc}
\usepackage[english]{babel}
%\usepackage{hyperref}
\usepackage[bottom = 2.5cm, top = 1.5cm, left = 1.5cm, right = 1.5cm]{geometry}
\usepackage{fancyhdr}
\renewcommand\familydefault{\sfdefault}
\linespread{1.4}
\setlength{\parindent}{4mm}
\setlength{\parskip}{1mm}
\pagestyle{fancy}
\fancyhf{}
\clubpenalty = 10000
\widowpenalty = 10000
\displaywidowpenalty = 10000
\fancyfoot{}
\cfoot{{\footnotesize \textnormal{Lorenz Leutgeb, Moritz Wanzenböck}}}
\renewcommand{\headrulewidth}{0pt}
\renewcommand{\footrulewidth}{0pt}
\let\origappendix\appendix
\begin{document}

\section*{Variable Neighborhood Search for the Partition Graph Coloring Problem}

\subsection*{Aim of Project}

The goal of this project is to adapt the metaheuristic approach \textit{Variable Neighborhood Search} (VNS) to the \textit{Partition Graph Coloring Problem} (PCP) of combinatorial optimization.

\subsubsection*{The Partition Graph Coloring Problem}

\paragraph{Definition}
Given an undirected graph $G = (V,\ E)$, where $V$ is the set of vertices of $G$, $E$ the set of edges (2-tupels of the form $(s,\ t)$, denoteing source and target vertex of each edge) connecting $V$ in $G$ and $V_1,\ V_2,\ V_3,\ \ldots,$ $V_k$ disjoint subsets of $V$ defined as $V_i \cap V_j = \emptyset \ \forall i,\ j = 1,\ \ldots,\ k$ with $i \not = j$, where $\bigcup_{i = 1}^k V_i = V$, called \textit{partitions}, \textit{clusters} or \textit{components}, the \textit{Partition Graph Coloring Problem} (PCP) consists of finding a set of vertices $V'$ such that $|V' \cap V_i| = 1,\ \forall\ i = 1,\ \ldots,\ k$ holds ($V'$ contains a vertex in every cluster) and the coloring $S$ of the graph $G' = (V', E')$ constructable with vertices induced by $V'$ and the implied subset of $E' = \{(i,\ j) \in E\ |\ (i \in V') \vee (j \in V') \}$ is minimal.

A solution is considered the set of \textit{representants} in $V'$ together with their coloring $S$.

\paragraph{Application}
Modern fiber optic networks using \textit{Wavelength Division Multiplexing} (WDM) can be modeled as instances of the PCP to solve the problem of assigning wavelengths for individual communication paths through the network. This way an ideal assignment of wavelengths can be found.

\subsection*{Materials and Methods}

\subsubsection*{Variable Neighborhood Search}
The PCP itself is non-deterministic polynomial-time hard (NP-hard) thus the time complexity of obtaining an optimal solution does not grow in polynonmial relation to the problem's size. Therefore exact methods to solve the problem get unusuable quickly because of the tremendously fast increasing computing time. A way to work around the problems of exact approaches are heuristical methods.

Heuristics do not necessarily find the optimal solution to a problem but in general are much faster than exact methods. The metaheurisitc VNS takes benefits from combining multiple heuristics in a clever way: Neighborhoods (small heuristics) are applied to an initial solution to improve it. By varying those procedures, optimization gets more effective, because local optimae of different heuristics can be designed in such a way that they rarely overlap. In case all neighborhoods are in a state of local optimality, the VNS triggers a shaking mechanism. Shaking means altering the current solution not only improving it but also accepting mutations that introduce worse solutions.

\subsection*{Observations and Conclusions}

The VNS approach promises good solutions to the PCP. We implemented the algorithm and computed results using 30\% to 40\% less colors on instances with 90 to 120 vertices, and 15\% on large instances with 500 vertices compared to the initial solution. To compare these results we used instances common with other publications on the PCP. Although this is already a signifacant figure, we are still working on improving our neighborhoods (heuristics).

\end{document}
