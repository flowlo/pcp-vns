\documentclass[paper = a4, fontsize = 10pt]{scrartcl}

%The written project report has five parts:
%	A typewritten presentation (or essay). Hand written presentations are not admissible. The presentation should describe the project. It may be accompanied by original illustrations (graphs, drawings and photographs).It may consist of up to a maximum of 10 pages of written text (A4 format; single sided; double spaced and unbound in a minimum character size of 10 point); It may be accompanied by up to a further 10 pages of illustrations (A4 format; single sided and unbound); No extra materials such as video tapes and diskettes can be accepted as part of the typewritten presentation
%	A one page scientific summary in English containing the most important points of the project (aim of project, materials and methods, observations and conclusions).
%	A clear, concise project title in English for the Contest Catalogue. This may be accompanied, if required, by the full scientific title.
%	The full original scientific title, in the original language.
%	A straightforward description of the project of not more than ten lines in simple English for publication in the Contest Catalogue. Contestants, through their National Organiser, must ensure that this brief project description is readily understandable by the reporting press, other media, and interested members of the wider public.

\usepackage[utf8]{inputenc}
\usepackage[english]{babel}
%\usepackage{hyperref}
\usepackage[bottom = 2.5cm, top = 1.5cm, left = 1.5cm, right = 1.5cm]{geometry}
\usepackage{fancyhdr}
\renewcommand\familydefault{\sfdefault}
\linespread{1.3}
%\setlength{\parindent}{4mm}
\setlength{\parindent}{0mm}
\setlength{\parskip}{1mm}
\pagestyle{fancy}
\fancyhf{}
\clubpenalty = 10000
\widowpenalty = 10000
\displaywidowpenalty = 10000
\fancyfoot{}
\cfoot{{\footnotesize \textnormal{Lorenz Leutgeb, Moritz Wanzenböck}}}
\renewcommand{\headrulewidth}{0pt}
\renewcommand{\footrulewidth}{0pt}
\let\origappendix\appendix


\begin{document}

\section*{Variable Neighborhood Search for the Partition Graph Coloring Problem}

\subsection*{Aim of Project}

%The goal of this project is to adapt the metaheuristic approach \textit{Variable Neighborhood Search} (VNS) to the \textit{Partition Graph Coloring Problem} (PCP) of combinatorial optimization.
The goal of this project is to distribute communication demands known in advance in large scale computer networks. This is done to allow for additional communication, thus avoiding or at least delaying the extension of these networks. This problem is transformed into the \textit{Partition Graph Coloring Problem} (PCP), a combinatorial optimization problem, which is solved by a metaheuristic called \textit{Variable Neighborhood Search} (VNS).


\subsubsection*{The Partition Graph Coloring Problem}

\paragraph{Definition}
Given an undirected graph $G = (V,\ E)$, where $V$ is the set of vertices of $G$, $E$ denotes the set of undirected edges connecting the vertices of $V$ in $G$, as well as $k$ disjoint subsets of $V$, namely $V_1,\ V_2,\ \ldots,$ $V_k$, $V_i \cap V_j = \emptyset \ \forall i,\ j = 1,\ \ldots,\ k$ with $i \not = j$, where $\bigcup_{i = 1}^k V_i = V$ (called \textit{partitions}, \textit{clusters} or \textit{components}), the \textit{Partition Graph Coloring Problem} (PCP) is to find a set of vertices $V'$ such that $|V' \cap V_i| = 1,\ \forall\ i = 1,\ \ldots,\ k$ ($V'$ contains exactly one vertex of each cluster) and the coloring $S$ of the graph $G' = (V', E')$ ($E'$ being the set of edges induced by $V'$, i.e.\ $E' = \{(a,\ b) \in E\ |\ a \in V' \vee b \in V' \}$) is minimal.

A solution is considered the set of \textit{representatives}, i.e.\ the vertices in $V'$, together with their coloring $S$ (minimal number of colors, where two connected vertices in $G'$ must have assigned different ones).

\paragraph{Application}
Modern fiber optic networks using \textit{Wavelength Division Multiplexing} (WDM) can be modeled as instances of the PCP to solve the problem of assigning wavelengths for individual communication paths through the network. This way an ideal assignment of wavelengths can be found.

\subsection*{Materials and Methods}

\subsubsection*{Variable Neighborhood Search}
The PCP is a non-deterministic polynomial-time hard (NP-hard) optimization problem, thus the time complexity of obtaining an optimal solution is not polynomially related to the size of the problem instance. Therefore exact methods to solve the PCP can only be applied to relatively small instances due to the tremendously fast increase of computing time. A way to work around the problems of exact approaches is to use heuristic methods.

Heuristics do not necessarily find the optimal solution to a problem but in general are much faster than exact approaches. The metaheuristic VNS takes advantage from combining multiple heuristics in a clever way: Neighborhoods (small heuristics to slightly modify a solution) are applied to an initial solution to incrementally improve it. Simple heuristics often get stuck in a local optimum (the solution can no longer be improved by a small step), by sophistically combining different heuristics the occurrence of such a situation can be reduced although not completely avoided. In case all neighborhoods are caught in a local optimum, the VNS triggers a shaking mechanism. Shaking means altering the current solution not with the aim to improve it but to escape the current local optimum, thus temporary accepting a worse solution.

\subsection*{Observations and Conclusions}

The VNS approach already leads to good results to the PCP. We implemented the proposed algorithm and initial solutions computed by a construction heuristic could be improved by 30\% to 40\% on instances with 90 to 120 vertices, and 15\% on large instances with 500 vertices. These instances are taken from the literature to make the results comparable to other publications in this field. Although these results are promising we are still working on improving our (meta-) heuristic framework, especially the design of our neighborhoods.

\end{document}
